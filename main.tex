\documentclass[11pt,a4paper]{article}
\usepackage[utf8]{inputenc}
\usepackage[margin=1in]{geometry}
\usepackage{color}
\usepackage{graphicx}
\usepackage{array}
\usepackage{verbatim}
\usepackage{caption}
\usepackage{subcaption}
\usepackage{amsmath,amsthm,amsfonts,amssymb,latexsym}
\usepackage{bbm}
\usepackage{setspace}
\usepackage{xparse}
\usepackage{epstopdf}
\usepackage{pgf}
\usepackage[colorlinks=true,citecolor=blue]{hyperref}
\usepackage[nameinlink,capitalise]{cleveref}

\usepackage{tikz}
\usepackage{tikz-cd}
\usepackage{pgfplotstable}
\pgfplotsset{compat=1.14}
\usetikzlibrary{patterns}
\usetikzlibrary{calc}
\usetikzlibrary{angles}
\usetikzlibrary{quotes}
\usetikzlibrary{external}

\onehalfspacing
% \setlength{\parskip}{6pt}

\DeclareDocumentCommand\abs{s m} {\IfBooleanTF{#1}{\left|#2\right|}{\left|#2\right|}}
\DeclareDocumentCommand\cont{o m o} {C\IfNoValueF{#1}{^{#1}}(#2\IfNoValueF{#3}{;#3})}
\DeclareDocumentCommand\contc{o m o} {C_c\IfNoValueF{#1}{^{#1}}(#2\IfNoValueF{#3}{;#3})}
\DeclareDocumentCommand\sobolev{m m o} {H^{#1}(#2 \IfNoValueF{#3}{,#3})}
\DeclareDocumentCommand\lp{m m o} {L^{#1}\left(#2 \IfNoValueF{#3}{,#3}\right)}
\DeclareDocumentCommand\norm{s m o} {\IfBooleanTF{#1}{\|#2\|}{\left\|#2\right\|}\IfNoValueF{#3}{_{#3}}}
\DeclareDocumentCommand\seminorm{m o o} {\left|#1\right|\IfNoValueF{#2}{_{#2 \IfNoValueF{#3}{,#3}}}}
\DeclareDocumentCommand\ip{s m m o} {\IfBooleanTF{#1}{\langle #2,#3 \rangle}{\left\langle #2,#3 \right\rangle}\IfNoValueF{#4}{_{#4}}}
\DeclareDocumentCommand\dup{m m o} {\left\langle{#1,#2}\right\rangle\IfNoValueF{#3}{_{#3', #3}}}
\DeclareDocumentCommand\gaussian{O{0} O{I}} {g_{#1, #2}}
\DeclareDocumentCommand\littleo{s o m} {o\IfNoValueF{#2}{_{#2}}\IfBooleanTF{#1}{(#3)}{\left(#3\right)}}
\DeclareDocumentCommand\bigo{s o m} {\mathcal O\IfNoValueF{#2}{_{#2}}\IfBooleanTF{#1}{(#3)}{\left(#3\right)}}


\DeclareMathOperator*{\argmax}{arg\,max}
\DeclareMathOperator*{\argmin}{arg\,min}
\DeclareMathOperator*{\re}{Re}
\DeclareMathOperator*{\trace}{tr}
\DeclareMathOperator{\Span}{span}
\DeclareMathOperator{\sym}{sym}
\DeclareMathOperator{\sign}{sign}
\DeclareMathOperator{\diag}{diag}
\DeclareMathOperator{\e}{e}
\DeclareMathOperator{\id}{id}
\DeclareMathOperator{\offdiag}{offdiag}

\newcommand{\revision}[1]{\textcolor{blue}{#1}}
\renewcommand{\revision}[1]{#1}
\newcommand{\gab}[1]{\textcolor{darkgreen}{#1}}
\newcommand{\commut}[2]{[#1, #2]}
\newcommand{\correlation}[1]{\left< #1 \right>}
\newcommand{\dummy}{\,\cdot\,}
\newcommand{\expect}[0]{\mathbf{E}}
\newcommand{\iip}[2]{\left(\!\left(#1, #2\right)\!\right)}
\newcommand{\nat}{\mathbf N}
\newcommand{\poly}{\mathbf P}
\newcommand{\real}{\mathbf R}
\newcommand{\integer}{\mathbf Z}
\newcommand{\torus}{\mathbf T}
\newcommand{\grad}{\boldsymbol \nabla}
\newcommand{\hess}{\nabla^2}
\newcommand{\vect}[1]{\boldsymbol{\mathbf #1}}
\newcommand{\mat}[1]{\vect #1}
\renewcommand{\det}[1]{\mathrm{det} \left( #1 \right)}
\renewcommand{\d}{\mathrm d}
\renewcommand{\t}{\mathsf T}
% \renewcommand{\t}{t}

\makeatletter
\DeclareDocumentCommand \derivative{s m o m}{%
    \def\@der{\IfBooleanTF{#1}{\mathrm{d}}{\partial}}
    \def\@default{%
        \mathchoice{%
                \frac{%
                    \@der\ifnum\pdfstrcmp{#2}{1}=0\else^{#2}\fi {\IfNoValueTF{#3}{}{#3}}
                }{%
                    \@for\@token:={#4}\do{\@der \@token}
                }
            } {%
                \@for\@token:={#4}\do{\@der_\@token} \IfNoValueTF{#3}{}{#3}
            } {} {}
    }
    \IfBooleanTF{#1}{\IfNoValueTF{#3}{\@default}{%
                #3%
                \ifnum\pdfstrcmp{#2}{1}=0'\else%
                \ifnum\pdfstrcmp{#2}{2}=0''\else%
                \ifnum\pdfstrcmp{#2}{3}=0^{(3)}\else%
                \ifnum\pdfstrcmp{#2}{4}=0^{(4)}\else%
                \ifnum\pdfstrcmp{#2}{5}=0^{(5)}\else%
                ^{(#2)}\fi\fi\fi\fi\fi
            }
        }{\@default}
}
\makeatother

\definecolor{darkred}{rgb}{.5,0,0}
\definecolor{darkgreen}{rgb}{0,.5,0}
\definecolor{darkblue}{rgb}{0,0,.5}
\newcommand{\red}[1]{\textcolor{darkred}{#1}}
\newcommand{\green}[1]{\textcolor{darkgreen}{#1}}

\theoremstyle{plain}
\newtheorem{assumption}{Assumption}[section]
\newtheorem{lemma}{Lemma}[section]
\newtheorem{corollary}{Corollary}[section]
\newtheorem{theorem}{Theorem}[section]
\newtheorem{proposition}{Proposition}[section]
\newtheorem{result}{Result}[section]
\newtheorem{remark}{Remark}[section]
\numberwithin{equation}{section}

\newcounter{urbainCounter}
\newcommand{\urbain}[1]{\stepcounter{urbainCounter}\red{\arabic{urbainCounter}.} \green{#1}}
\crefname{equation}{}{}
\crefname{paragraph}{\S\!}{\S}
% \crefname{figure}{Figure}{Figures}
% \crefname{section}{Section}{Sections}

\newcommand{\email}[1]{\href{#1}{#1}}
\newcommand{\orcidcolor}{ORC\textcolor{orcidlogocol}{ID}}
\newcommand{\orcid}[1]{\href{https://orcid.org/#1}{\includegraphics[width=.4cm]{z_orcid.pdf}}}

%---------------- GABRIEL ------------
\usepackage{enumerate}
\newcommand{\eps}{\varepsilon}
\newcommand{\dps}{\displaystyle}
\newcommand{\cX}{\mathcal{X}}
\newcommand{\ri}{\mathrm{i}}
\renewcommand{\leq}{\leqslant}
\renewcommand{\geq}{\geqslant}
\renewcommand{\le}{\leqslant}
\renewcommand{\ge}{\geqslant}
% \usepackage{todonotes}
\usepackage{mathrsfs}

% BODY {{{1
\date{\today}
\title{Langevin dynamics in the underdamped regime: effective diffusion and variance reduction }
\author{%
  % G.A. Pavliotis\thanks{Department of Mathematics, Imperial College London (\email{g.pavliotis@imperial.ac.uk})}%
  % \hspace{2mm}\orcid{0000-0002-3468-9227}%
  % \and G. Stoltz\thanks{CERMICS, \'Ecole des Ponts, France \& MATHERIALS, Inria Paris (\email{gabriel.stoltz@enpc.fr})}
  % \hspace{2mm}\orcid{0000-0002-2797-5938}%
  % \and U. Vaes\thanks{Department of Mathematics, Imperial College London (until October 2020) and MATHERIALS, Inria Paris (since November 2020) (\email{urbain.vaes@inria.fr})}%
  % \hspace{2mm}\orcid{0000-0002-7629-7184}%
}

\begin{document}
\maketitle
Let us first consider the Langevin dynamics in one dimension:
\begin{align*}
    \d q &= p \, \d t, \\
    \d p &= - \derivative*{1}[V]{q}(q) \, \d t - \gamma p \, \d t + \sqrt{2 \gamma \beta^{-1}} \d W_t.
\end{align*}
The generator of the associated Markov semigroup is given by
\[
    \mathcal L = p \derivative{1}{q} - \derivative*{1}[V]{q}(q) \derivative{1}{p} + \gamma \left( - p \derivative{1}{p} + \beta^{-1} \derivative{2}{p^2} \right).
\]
Let $\phi$ denote the solution to
\[
    - \mathcal L \phi = p,
\]
and suppose that $\psi$ denote an approximation of $\phi$.
By It\^o's formula, it holds
\begin{align*}
    \phi(q_t, p_t) - \phi(q_0, p_0) &= - \int_{0}^{t} p_s \, \d s + \sqrt{2 \gamma \beta^{-1}} \int_{0}^{t} \derivative{1}[\phi]{p}(q_s, p_s) \, \d W_s \\
    \psi(q_t, p_t) - \psi(q_0, p_0) &= \int_{0}^{t} (\mathcal L \psi) (q_s, p_s) \, \d s + \sqrt{2 \gamma \beta^{-1}} \int_{0}^{t} \derivative{1}[\psi]{p}(q_s, p_s) \, \d W_s.
\end{align*}
Let also $u(t)$ and $v(t)$ denote the random variables
\[
    u(t) = \frac{\abs{q_t - q_0}^2}{2t}
\]
and
\begin{align*}
    v(t) &= \int_{\torus \times \real} \abs{\derivative{1}[\psi]{p}}^2 \, \d \mu + \frac{1}{2t} \left( \abs{q_t - q_0}^2 - \abs{\psi(q_t, p_t) - \psi(q_0, p_0) - \sqrt{2 \gamma \beta^{-1}} \int_{0}^{t} \derivative{1}[\psi]{p}(q_s, p_s) \, \d W_s}^2\right) \\
         &=: D_{\psi}+ \frac{1}{2t} \left( \abs{q_t - q_0}^2 - \abs{\xi_t}^2\right).
\end{align*}
It holds
\[
    \lim_{t \to \infty} \expect \left( u(t) \right) = \lim_{t \to \infty} \expect \left( v(t) \right)
    = \gamma \beta^{-1} \int_{\torus \times \real} \abs{\derivative{1}[\phi]{p}}^2 \, \d \mu
    = \int_{\torus \times \real} \phi \, p \, \d \mu =: D_{\phi}.
\]
so both $u(t)$ and $v(t)$ are asymptotically unbiased estimators of $D_{\phi}$.
For finite times, using It\^o's isometry and assuming stationary initial conditions, we obtain
\begin{align*}
    \expect \left( u(t) \right)
    &= \frac{1}{2t} \expect \left( \abs{\phi(t) - \phi(0)}^2 \right) + \expect \abs{ \sqrt{\frac{\gamma \beta^{-1}}{t}} \int_{0}^{t} \derivative{1}[\psi]{p}(q_s, p_s) \, \d W_s}^2 + \expect \left( \frac{\sqrt{2 \gamma \beta^{-1}}}{t} (\phi(t) - \phi(0))  \int_{0}^{t} \derivative{1}[\psi]{p}(q_s, p_s) \, \d W_s \right) \\
    &= \frac{1}{2t} \expect \left( \abs{\phi(t) - \phi(0)}^2\right) + D_{\phi} + R(t),
\end{align*}
where the remainder term $R(t)$ satisfies
\begin{align*}
    R(t) \leq \varepsilon D_{\phi} + \frac{1}{4 \varepsilon} \frac{1}{2t} \expect \left( \abs{\phi(t) - \phi(0)}^2\right) \qquad \forall \varepsilon > 0.
\end{align*}
Taking $\varepsilon = 1/\sqrt{\gamma t}$, we deduce
\[
    \abs{\expect \left( u(t) \right) - D_{\phi}}
    \leq \frac{D_{\phi}}{\sqrt{\gamma t}} + \left( \frac{1}{2t} + \frac{\sqrt{\gamma}}{8\sqrt{t}} \right)\expect \left( \abs{\phi(t) - \phi(0)}^2\right).
\]
$\rightarrow$ The relative error scales as \( 1 / \sqrt{\gamma t}\).
Applying the same reasoning to $v(t)$, together with the inequality
\[
    \abs{x_1 y_1 - x_2 y_2} \leq \frac{y_1^2}{4 \varepsilon_1} + \varepsilon_1 \abs{x_1 - x_2}^2 + \frac{x_2^2}{4 \varepsilon_2} + \varepsilon_2 \abs{y_1 - y_2}^2,
\]
we obtain
\begin{align*}
    \abs{\expect \left( v(t) \right) - D_{\phi}}
    &\leq \frac{D_{\phi}}{\sqrt{t}} + \frac{1}{2t} \expect\left( \abs{\phi(t) - \phi(0)}^2 - \abs{\psi(t) - \psi(0)}^2 \right) \\
    &\quad + \gamma \beta^{-1} \, \frac{\varepsilon_1}{\sqrt{t}}\int \left|\derivative{1}[\psi]{p} - \derivative{1}[\phi]{p} \right|^2 \d \mu
    + \frac{\varepsilon_2}{\sqrt{t}} \expect |\phi(t) - \phi(0) - \psi(t) + \psi(0)|^2 \\
    &\quad + \frac{1}{8 \varepsilon_1 \sqrt{t}} \expect \left( |\phi(t) - \phi(0)|^2 \right) + \frac{1}{8 \varepsilon_2 \sqrt{t}} D_{\psi}.
\end{align*}
$\rightarrow$ better constant on the right-hand side when $\psi \approx \phi$,
and when $\psi = 0$ we recover the previous bound.
For the variance, we can obtain a crude bound using
\begin{align*}
    \mathrm{Var} (v(t))
    &\leq \expect \left( \abs{v(t) - D_{\psi}}^2 \right)
    = \frac{1}{4 t^2} \expect \big( |q_t - q_0 - \xi_t|^2 |q_t - q_0 + \xi_t|^2 \big) \\
    &\leq \frac{1}{4 t^2} \sqrt{\expect \left(  |q_t - q_0 - \xi_t|^4 \right)} \sqrt{\expect \left( |q_t - q_0 + \xi_t|^4 \right)}
\end{align*}
We now bound \red{signs probably wrong, constants 1}
\begin{align*}
    \expect \left( |q_t - q_0 + \xi_t|^4 \right)
    &= \expect \left( \abs{\phi_t - \phi_0 - \psi_t + \psi_0 + I_{\psi} - I_{\phi}}^4 \right) \\
    &\leq C \left( \expect \left( \abs{\phi_t - \psi_t}^4 \right) + \expect \left( \abs{\phi_0 - \psi_0}^4 \right) + \expect \left( \abs{I_{\psi} - I_{\phi}}^4 \right) \right).
\end{align*}
The first two terms are bounded by $\norm{\phi - \psi}_{L^4(\mu)}$.
Using a moment inequality for It\^o integrals, the last term can be bounded as
\[
    \expect \left( \abs{I_{\psi} - I_{\phi}}^4 \right) \leq \expect \int_{0}^{t} \abs{\derivative{1}[\phi]{p}(q_s, p_s) - \derivative{1}[\psi]{p}(q_s, p_s)}^4 \, \d s = \int \abs{\derivative{1}[\phi]{p} - \derivative{1}[\psi]{p}}^4 \d \mu.
\]
\end{document}
