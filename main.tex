\documentclass[11pt,a4paper]{article}
\usepackage[utf8]{inputenc}
\usepackage[margin=1in]{geometry}
\usepackage{color}
\usepackage{graphicx}
\usepackage{microtype}
\usepackage{array}
\usepackage{verbatim}
\usepackage{caption}
\usepackage{subcaption}
\usepackage{amsmath,amsthm,amsfonts,amssymb,latexsym}
\usepackage{bbm}
\usepackage{setspace}
\usepackage{xparse}
\usepackage{epstopdf}
\usepackage{pgf}
\usepackage[colorlinks=true,citecolor=blue]{hyperref}
\usepackage[nameinlink,capitalise]{cleveref}

\usepackage[style=trad-abbrv,doi=false,url=false,isbn=false,backend=biber]{biblatex}
\DeclareFieldFormat{volume}{volume \textbf{#1}}
\DeclareFieldFormat[article]{volume}{\textbf{#1}}
\addbibresource{main.bib}

\usepackage{tikz}
\usepackage{tikz-cd}
\usepackage{pgfplotstable}
\pgfplotsset{compat=1.14}
\usetikzlibrary{patterns}
\usetikzlibrary{calc}
\usetikzlibrary{angles}
\usetikzlibrary{quotes}
\usetikzlibrary{external}

\onehalfspacing
% \setlength{\parskip}{6pt}

\DeclareDocumentCommand\abs{s m} {\IfBooleanTF{#1}{\left|#2\right|}{\left|#2\right|}}
\DeclareDocumentCommand\cont{o m o} {C\IfNoValueF{#1}{^{#1}}(#2\IfNoValueF{#3}{;#3})}
\DeclareDocumentCommand\contc{o m o} {C_c\IfNoValueF{#1}{^{#1}}(#2\IfNoValueF{#3}{;#3})}
\DeclareDocumentCommand\sobolev{m m o} {H^{#1}(#2 \IfNoValueF{#3}{,#3})}
\DeclareDocumentCommand\lp{m m o} {L^{#1}\left(#2 \IfNoValueF{#3}{,#3}\right)}
\DeclareDocumentCommand\norm{s m o} {\IfBooleanTF{#1}{\|#2\|}{\left\|#2\right\|}\IfNoValueF{#3}{_{#3}}}
\DeclareDocumentCommand\seminorm{m o o} {\left|#1\right|\IfNoValueF{#2}{_{#2 \IfNoValueF{#3}{,#3}}}}
\DeclareDocumentCommand\ip{s m m o} {\IfBooleanTF{#1}{\langle #2,#3 \rangle}{\left\langle #2,#3 \right\rangle}\IfNoValueF{#4}{_{#4}}}
\DeclareDocumentCommand\dup{m m o} {\left\langle{#1,#2}\right\rangle\IfNoValueF{#3}{_{#3', #3}}}
\DeclareDocumentCommand\gaussian{O{0} O{I}} {g_{#1, #2}}
\DeclareDocumentCommand\littleo{s o m} {o\IfNoValueF{#2}{_{#2}}\IfBooleanTF{#1}{(#3)}{\left(#3\right)}}
\DeclareDocumentCommand\bigo{s o m} {\mathcal O\IfNoValueF{#2}{_{#2}}\IfBooleanTF{#1}{(#3)}{\left(#3\right)}}


\DeclareMathOperator*{\argmax}{arg\,max}
\DeclareMathOperator*{\argmin}{arg\,min}
\DeclareMathOperator*{\re}{Re}
\DeclareMathOperator*{\trace}{tr}
\DeclareMathOperator{\Span}{span}
\DeclareMathOperator{\sym}{sym}
\DeclareMathOperator{\sign}{sign}
\DeclareMathOperator{\diag}{diag}
\DeclareMathOperator{\e}{e}
\DeclareMathOperator{\id}{id}
\DeclareMathOperator{\offdiag}{offdiag}

\newcommand{\revision}[1]{\textcolor{blue}{#1}}
\renewcommand{\revision}[1]{#1}
\newcommand{\gab}[1]{\textcolor{darkgreen}{#1}}
\newcommand{\commut}[2]{[#1, #2]}
\newcommand{\correlation}[1]{\left< #1 \right>}
\newcommand{\dummy}{\,\cdot\,}
\newcommand{\expect}[0]{\mathbf{E}}
\newcommand{\var}[0]{\mathbf{V}}
\newcommand{\iip}[2]{\left(\!\left(#1, #2\right)\!\right)}
\newcommand{\nat}{\mathbf N}
\newcommand{\poly}{\mathbf P}
\newcommand{\real}{\mathbf R}
\newcommand{\integer}{\mathbf Z}
\newcommand{\torus}{\mathbf T}
\newcommand{\grad}{\nabla}
\newcommand{\hess}{\nabla^2}
\newcommand{\vect}[1]{\boldsymbol{\mathbf #1}}
\newcommand{\mat}[1]{\vect #1}
\renewcommand{\det}[1]{\mathrm{det} \left( #1 \right)}
\renewcommand{\d}{\mathrm d}
\renewcommand{\t}{\mathsf T}
% \renewcommand{\t}{t}

\makeatletter
\DeclareDocumentCommand \derivative{s m o m}{%
    \def\@der{\IfBooleanTF{#1}{\mathrm{d}}{\partial}}
    \def\@default{%
        \mathchoice{%
                \frac{%
                    \@der\ifnum\pdfstrcmp{#2}{1}=0\else^{#2}\fi {\IfNoValueTF{#3}{}{#3}}
                }{%
                    \@for\@token:={#4}\do{\@der \@token}
                }
            } {%
                \@for\@token:={#4}\do{\@der_\@token} \IfNoValueTF{#3}{}{#3}
            } {} {}
    }
    \IfBooleanTF{#1}{\IfNoValueTF{#3}{\@default}{%
                #3%
                \ifnum\pdfstrcmp{#2}{1}=0'\else%
                \ifnum\pdfstrcmp{#2}{2}=0''\else%
                \ifnum\pdfstrcmp{#2}{3}=0^{(3)}\else%
                \ifnum\pdfstrcmp{#2}{4}=0^{(4)}\else%
                \ifnum\pdfstrcmp{#2}{5}=0^{(5)}\else%
                ^{(#2)}\fi\fi\fi\fi\fi
            }
        }{\@default}
}
\makeatother

\definecolor{darkred}{rgb}{.5,0,0}
\definecolor{darkgreen}{rgb}{0,.5,0}
\definecolor{darkblue}{rgb}{0,0,.5}
\newcommand{\red}[1]{\textcolor{darkred}{#1}}
\newcommand{\green}[1]{\textcolor{darkgreen}{#1}}

\theoremstyle{plain}
\newtheorem{assumption}{Assumption}[section]
\newtheorem{lemma}{Lemma}[section]
\newtheorem{corollary}{Corollary}[section]
\newtheorem{theorem}{Theorem}[section]
\newtheorem{proposition}{Proposition}[section]
\newtheorem{result}{Result}[section]
\newtheorem{remark}{Remark}[section]
\newtheorem{example}{Example}[section]
\numberwithin{equation}{section}

\newcounter{urbainCounter}
\newcommand{\urbain}[1]{\stepcounter{urbainCounter}\red{\arabic{urbainCounter}.} \green{#1}}
\crefname{equation}{}{}
\crefname{paragraph}{\S\!}{\S}
% \crefname{figure}{Figure}{Figures}
% \crefname{section}{Section}{Sections}

\newcommand{\email}[1]{\href{#1}{#1}}
\newcommand{\orcidcolor}{ORC\textcolor{orcidlogocol}{ID}}
\newcommand{\orcid}[1]{\href{https://orcid.org/#1}{\includegraphics[width=.4cm]{z_orcid.pdf}}}

%---------------- GABRIEL ------------
\usepackage{enumerate}
\newcommand{\eps}{\varepsilon}
\newcommand{\dps}{\displaystyle}
\newcommand{\cX}{\mathcal{X}}
\newcommand{\ri}{\mathrm{i}}
\renewcommand{\leq}{\leqslant}
\renewcommand{\geq}{\geqslant}
\renewcommand{\le}{\leqslant}
\renewcommand{\ge}{\geqslant}
% \usepackage{todonotes}
\usepackage{mathrsfs}

% BODY {{{1
\date{\today}
\title{Mobility estimation for Langevin dynamics using control variates}
\author{%
  % G.A. Pavliotis\thanks{Department of Mathematics, Imperial College London (\email{g.pavliotis@imperial.ac.uk})}%
  % \hspace{2mm}\orcid{0000-0002-3468-9227}%
  % \and G. Stoltz\thanks{CERMICS, \'Ecole des Ponts, France \& MATHERIALS, Inria Paris (\email{gabriel.stoltz@enpc.fr})}
  % \hspace{2mm}\orcid{0000-0002-2797-5938}%
  % \and U. Vaes\thanks{Department of Mathematics, Imperial College London (until October 2020) and MATHERIALS, Inria Paris (since November 2020) (\email{urbain.vaes@inria.fr})}%
  % \hspace{2mm}\orcid{0000-0002-7629-7184}%
}

\begin{document}
\maketitle

\begin{abstract}
    The scaling of the mobility of multi-dimensional Langevin dynamics in a periodic potential as the friction vanishes is not well understood for non-separable potentials.
    Theoretical results are lacking,
    and numerical calculation of the mobility in the underdamped regime is challenging because
    the computational cost required to achieve a given relative accuracy is inversely proportional to the friction coefficient.
    In this note, we propose a new variance-reduction method based on control variates for efficiently estimating the mobility of Langevin-type dynamics.
    We provide bounds on the bias and variance of the proposed estimator,
    and we illustrate its efficacy through numerical experiments,
    first in simple one-dimensional settings
    and then for two-dimensional Langevin dynamics.
    Our results corroborate previous numerical evidence that
    the mobility scales as~$\gamma^{-\alpha}$ in the low friction regime for a simple non-separable potential.
\end{abstract}

\section{Introduction}%
Langevin-type dynamics are Newtonian atomistic models for the evolution of a system of particles interacting with an environment at fixed temperature.
They are widely used for the calculation of macroscopic properties of matter in molecular simulation.
The standard Langevin dynamics, sometimes called underdamped Langevin dynamics,
in a periodic potential $V$ reads
\begin{subequations}
\label{eq:langevin}
\begin{align}
    \label{eq:langevin_q}
    \d q_t &= M^{-1} p_t \, \d t, \\
    \label{eq:langevin_p}
    \d p_t &= - \grad V(q_t) \, \d t - \gamma M^{-1} \, p_t \, \d t + \sqrt{2 \gamma \beta^{-1}} \d W_t.
\end{align}
\end{subequations}
Here $q_t \in \torus^d$ and $p_t \in \real^d$ are the position and momentum variables,
with~$\torus^d = \real^d / 2\pi \integer^d$ the $d$-dimensional torus with period $2 \pi$.
The parameter $\gamma$ is the friction,
$M$ is the mass matrix,
and~$W_t$ is a standard $d$-dimensional Brownian motion.
For simplicity we assume that $M = m I_d$,
where $I_d \in \real^{d \times d}$ is the identity matrix.
In this case $(\widetilde q_t, \widetilde p_t) := (q_{\sqrt{m} t}, m^{-1/2} p_{\sqrt{m} t})$
is a weak solution of~\eqref{eq:langevin} with $m = 1$ and $\gamma$ replaced by $\gamma/\sqrt{m}$,
so to further simplify we assume $m = 1$;
see~\cite[Section 2.2.4]{MR2681239} for a more detailed motivation of this simplification.
% keeping in mind that results obtained in the limit;


The generator of the Markov semigroup associated with~\eqref{eq:langevin} is given by
\[
    \mathcal L = p \derivative{1}{q} - \derivative*{1}[V]{q}(q) \derivative{1}{p} + \gamma \left( - p \derivative{1}{p} + \beta^{-1} \derivative{2}{p^2} \right).
\]
Let $\phi$ denote the solution to
\[
    - \mathcal L \phi = p,
\]
and suppose that $\psi$ denote an approximation of $\phi$.
By It\^o's formula, it holds
\begin{align}
    \notag
    \phi(q_t, p_t) - \phi(q_0, p_0) &= - \int_{0}^{t} p_s \, \d s + \sqrt{2 \gamma \beta^{-1}} \int_{0}^{t} \derivative{1}[\phi]{p}(q_s, p_s) \, \d W_s \\
    \label{eq:ito_control}
    \psi(q_t, p_t) - \psi(q_0, p_0) &= \int_{0}^{t} (\mathcal L \psi) (q_s, p_s) \, \d s + \sqrt{2 \gamma \beta^{-1}} \int_{0}^{t} \derivative{1}[\psi]{p}(q_s, p_s) \, \d W_s.
\end{align}
Let also $u(t)$ and $v(t)$ denote the random variables
\begin{equation}
    \label{eq:simple_estimator}
    u(t) = \frac{\abs{q_t - q_0}^2}{2t}
\end{equation}
and
\begin{align*}
    v(t) &= D_{\psi}+ \frac{1}{2t} \left( \abs{q_t - q_0}^2 - \abs{\xi_t}^2\right)
\end{align*}
where $D_{\psi} := \gamma \beta^{-1} \int_{\torus \times \real} \abs{\derivative{1}[\psi]{p}}^2 \, \d \mu$ and
 \[
    \xi_t = \psi(q_t, p_t) - \psi(q_0, p_0) - \sqrt{2 \gamma \beta^{-1}} \int_{0}^{t} \derivative{1}[\psi]{p}(q_s, p_s) \, \d W_s
 \]
It holds
\[
    \lim_{t \to \infty} \expect \bigl( u(t) \bigr) = \lim_{t \to \infty} \expect \bigl( v(t) \bigr)
    = \gamma \beta^{-1} \int_{\torus \times \real} \abs{\derivative{1}[\phi]{p}}^2 \, \d \mu
    = \int_{\torus \times \real} \phi \, p \, \d \mu =: D_{\phi}.
\]
so both $u(t)$ and $v(t)$ are asymptotically unbiased estimators of $D_{\phi}$.

\paragraph{Bias of the estimator.}%
We study in this section the bias of the estimators $u(t)$ and $v(t)$ under stationary initial conditions,
that is $(q_0, p_0) \sim \mu$.
In this case, it holds that
\begin{equation}
\label{eq:bias_without_control}
\begin{aligned}
    \expect \bigl(u(t)\bigr)
    &= \frac{1}{2t} \expect \left( \int_{0}^{t} p_s \, \d s \int_{0}^{t} p_u \, \d u \right)
    = \frac{1}{2t} \left( \int_{0}^{t} \int_{0}^{t} \expect (p_s p_u) \, \d s \, \d u \right) \\
    &= \frac{1}{2t} \left( \int_{0}^{t} \int_{0}^{t} \ip{\e^{\abs{s - u} \mathcal L}p}{p} \, \d s \, \d u \right)
    =  \int_{0}^{t} \ip{\e^{s \mathcal L}p}{p} \left(1 - \frac{s}{t}\right) \d s  \\
    &= \int_{0}^{\infty} \ip{\e^{s \mathcal L}p}{p}  \d s - \int_{0}^{t} \ip{\e^{s \mathcal L}p}{p} \min\left(1, \frac{s}{t}\right) \, \d s.
    % D_{\phi} - \int_{0}^{\infty} \min\left(1, \frac{s}{t}\right) C_{\phi}(s) \, \d s.
\end{aligned}
\end{equation}
The first term is the effective diffusion coefficient,
and the second term is the bias.
In order to obtain a bound on the bias,
a natural approach is to use a known bound on the Markov semigroup associated with Langevin dynamics~\cite{roussel2018spectral},
stating that
\begin{equation}
    \label{eq:decay_semigroup_general}
    \forall \gamma > 0, \qquad \forall t \geq 0, \qquad
    \norm*{ \e^{s \mathcal L} }[\mathcal B \left(L^2_0\left(\mu\right) \right)] \leq M \exp \bigl(- \lambda s \min(\gamma, \gamma^{-1}) \bigr)
\end{equation}
for appropriate constants $M > 0$ and $\hat \lambda > 0$.
An application of this bound gives
\(
    \ip{\e^{s \mathcal L}p}{p} \leq \norm*{\e^{s \mathcal L}p} \norm*{p} \leq \beta^{-1} \exp\bigl(- \lambda s \min(\gamma, \gamma^{-1})\bigr),
\)
which leads to the following estimate on the bias:
\begin{equation}
    \label{eq:bias}
    \abs{\expect_{\mu} u(t) - D}
    \leq \frac{1}{\beta t}\int_{0}^{\infty} \exp\bigl(- \lambda s \min(\gamma, \gamma^{-1})\bigr) s \, \d s
    = \frac{\max (\gamma^2, \gamma^{-2})}{\beta \lambda^2 t}.
\end{equation}
Since the effective diffusion coefficient scales as $\gamma^1$ in both the underdamped ($\gamma \to 0$) and overdamped limits ($\gamma \to \infty$)~\cite{MR2394704,MR2427108},
this estimate suggests that the relative bias of the estimator scales as $\max(\gamma^{-1}, \gamma^3) t$ and that,
consequently, the integration time should scale as $t \propto \max(\gamma^{-1}, \gamma^3)$ in order to meet a given threshold on the relative error.
It turns out that, in fact, the estimate~\eqref{eq:bias} is not optimal,
and it is sufficient to choose $t \propto \gamma$ in the overdamped limit.
We can derive a sharper estimate from the following result:
\begin{proposition}
    \label{proposition:semigroup_meanzero_observable}
    Assume that $f(q, p) = Q(q) P(p)$ for smooth functions $Q \in H^5(\nu)$ and $P \in H^2_0(\kappa)$.
    Then there exist constants $C$ and $\lambda$ independent of $\gamma$, $Q$ and $P$ such that
    \begin{equation}
        \label{eq:optimal_decay_correlation}
        \forall \gamma \geq 1, \qquad
        \forall t \geq 0, \qquad
        \abs{\ip{\e^{t \mathcal L}f}{h}}
        \leq C \, \norm{f}[5] \norm{h}  \left( \gamma^{-2} \e^{- \lambda \gamma^{-1} t} + \e^{-\lambda  \gamma t} \right).
    \end{equation}
\end{proposition}
Applying this result with $f(q, p) = p$, we obtain
\begin{equation*}
    \forall \gamma \geq 1, \qquad
    \abs{\expect_{\mu} u(t) - D}
    \leq \frac{C}{t}\int_{0}^{\infty} \left( \gamma^{-2} \e^{- \lambda \gamma^{-1} s} + \e^{-\lambda  \gamma s} \right)  s \, \d s
    \leq \frac{C}{t}.
\end{equation*}
The proof of \cref{proposition:semigroup_meanzero_observable} is postponed to \cref{sec:auxiliary_technical_results}.
In this section,
we motivate the result by scrutinizing two examples where
explicit expressions of the bias or of the velocity auto-correlation can be obtained:
constant potential and quadratic potential.
\begin{example}
    Consider the case where $V(q) = 0$.
    In this case, the solution to the Poisson equation $- \mathcal L \phi = p$ is given by $\phi(q, p) = \gamma^{-1} p$ and,
    applying Itô's formula to this function, we obtain
    (note that this equation can be derived directly from~\eqref{eq:langevin_p})
    \[
        \gamma^{-1}(p_t - p_0) = - \int_{0}^{t} p_s \, \d s + \sqrt{2 \gamma^{-1} \beta^{-1}} (W_t - W_0)
        = q_0 - q_t + \sqrt{2 \gamma^{-1} \beta^{-1}} W_t.
    \]
    We deduce, using the explicit solution to the Ornstein--Uhlenbeck equation satisfied by $p$, that
    \begin{align*}
        q_t - q_0
        &= - \gamma^{-1} \left( p_0 \left(\e^{-\gamma t} - 1\right) + \sqrt{2 \gamma \beta^{-1}}\int_{0}^{t} \e^{-\gamma (t - s)} \, \d W_s \right)
        + \sqrt{2 \gamma^{-1} \beta^{-1}} W_t \\
        &=  - \gamma^{-1} p_0 \left(\e^{-\gamma t} - 1\right) + \sqrt{2 \gamma^{-1} \beta^{-1}}\int_{0}^{t} \left(1 - \e^{-\gamma (t - s)}\right) \, \d W_s.
    \end{align*}
    Assuming $p_0 \sim \mathcal N(0, \beta^{-1})$,
    the right-hand side of this equation is a mean-zero Gaussian random variable and,
    using It\^o's isometry, we calculate that
    \[
        \frac{\expect \abs{q_t - q_0}^2}{2t} = \gamma^{-1} \beta^{-1} \left( 1 + \frac{1}{t \gamma} \left(\e^{-\gamma t} - 1\right) \right).
    \]
    In this example, the relative bias is bounded from by $(t \gamma)^{-1}$.
\end{example}

\begin{example}
    [Case of a quadratic potential]
    \label{example:quadratic}
    Consider the case of the quadratic confining potential $V(q) = \frac{k q^2}{2}$.
    In this case, the eigenfunctions of $\mathcal L$ are polynomials,
    with the linear ones and associated eigenvalues being
    \[
        v_{\pm}(q, p) =
        \left( \frac{\gamma \pm \sqrt{\gamma^2 - 4k}}{2} \right) q + p, \\
        \qquad
        \lambda_{\pm} = \frac{- \gamma \pm \sqrt{\gamma^2 - 4k}}{2}.
    \]
    The function $(q, p) \mapsto p$ is the following linear combination of these eigenfunctions:
    \[
        p =
        \left( \frac{-\gamma + \sqrt{\gamma^2 - 4k}}{2 \sqrt{\gamma^2 - 4k}} \right) v_+
        + \left( \frac{\gamma + \sqrt{\gamma^2 - 4k}}{2 \sqrt{\gamma^2 - 4k}} \right) v_-.
    \]
    Therefore, given that $\ip{v_+}{p} = \ip{v_-}{p} = \beta^{-1}$,
    the velocity autocorrelation function is
    \[
        \ip{\e^{t \mathcal L}p}{p} =
        \left( \frac{-\gamma + \sqrt{\gamma^2 - 4k}}{2 \beta \sqrt{\gamma^2 - 4k}} \right) \e^{\lambda_+ t} +
        \left( \frac{\gamma + \sqrt{\gamma^2 - 4k}}{2 \beta \sqrt{\gamma^2 - 4k}} \right) \e^{\lambda_- t} = T_1(t) + T_2(t).
    \]
    The velocity autocorrelation function has a similar form that that in \cref{proposition:semigroup_meanzero_observable}.
    Indeed the factor multiplying the first term scales as $\bigo{\gamma^{-2}}$
    and $\lambda_+ = \bigo {\gamma^{-1}}$ as $\gamma \to \infty$.
    Note that, in this non-periodic case, the effective diffusion coefficient is equal to zero.
    % it holds that $\int_{0}^{\infty} T_1(t) \, \d t = \bigo{\gamma^{-1}}$
    % because the factor multiplying the exponential scales as $\bigo {\gamma^{-2}}$.
\end{example}

Concerning the bias of estimator $v(t)$,
we have the following result.

\begin{proposition}
    [Bias of the estimator]
    Assume stationary initial conditions~$(q_0, p_0) \sim \mu$.
    Then there exists $C$ independent of $\psi$ and $\gamma$ such that
    \begin{align}
        \label{eq:basic_bound_bias}
        \forall t \geq 0, \quad \gamma > 0, \qquad
        \abs{\expect \bigl( v(t) \bigr) - D_{\phi}}
        \leq &\frac{C}{t} \, \max(\gamma^2, \gamma^{-2}) \norm{p + \mathcal L\psi}  \left(1 + \norm{\mathcal L \psi} \right).
    \end{align}
    If $\mathcal L \psi \in H^5(\mu)$,
    this estimate can be refined in the overdamped limit to
    \begin{align*}
        \forall t \geq 0, \quad \gamma \geq 1, \qquad
        \abs{\expect \bigl( v(t) \bigr) - D_{\phi}}
        &\leq C t^{-1}
            \norm{p + (\id - \Pi_p) \mathcal L \psi} \bigl(1 + \norm{\mathcal L \psi}[5] \bigr) \\
             &\quad + C t^{-1} \gamma^2 \norm{\Pi_p \mathcal L \psi} \norm{\mathcal L \psi}
    \end{align*}
\end{proposition}
\begin{proof}
Using Itô's formula~\eqref{eq:ito_control} and the same reasoning as in~\eqref{eq:bias_without_control},
we obtain
\begin{align*}
    \expect_{\mu} v(t)
    &= D_{\psi} + \left( \frac{1}{2t} \right) \expect_{\mu} \biggl( \abs{q_t - q_0}^2 - \biggl| \int_0^t {\mathcal L \psi}(q_s, p_s) \, \d s \biggr|^2 \biggr) \\
    &= D_{\psi} +  \frac{1}{2t}  \int_{0}^{t} \Bigl( \ip{\e^{s \mathcal L}p}{p} - \ip{\e^{s \mathcal L} \mathcal L \psi}{\mathcal L \psi} \Bigr) \min\left(1, \frac{s}{t}\right) \d s \\
    &= D_{\phi} - \int_{0}^{\infty} \min\left(1, \frac{s}{t}\right) \Bigl( \ip{\e^{s \mathcal L}p}{p} - \ip{\e^{s \mathcal L} \mathcal L \psi}{\mathcal L \psi} \Bigr) \, \d s.
\end{align*}
In order to obtain~\eqref{eq:basic_bound_bias}, we write
\begin{align*}
     \ip{\e^{s \mathcal L}p}{p} - \ip{\e^{s \mathcal L} \mathcal L \psi}{\mathcal L \psi}
    &= \ip{p + \mathcal L \psi}{\e^{s \mathcal L} p - \e^{s \mathcal L^*} \mathcal  L\psi} \\
    &\leq \norm{p + \mathcal L \psi}
    \left( \norm*{\e^{s \mathcal L}}[\mathcal B\left(L^2_0(\mu) \right)] \norm{p} + \norm*{\e^{s \mathcal L^*}}[\mathcal B\left(L^2_0(\mu) \right)] \norm{\mathcal L \psi} \right) \\
    &\leq M \e^{- \lambda \min(\gamma, \gamma^{-1}) s} \norm{\mathcal L\psi + p}  \left(\beta^{-1/2} + \norm{\mathcal L \psi} \right).
\end{align*}

To obtain the refined bound in the overdamped limit,
we write
\begin{align*}
    \ip{\e^{s \mathcal L}p}{p} - \ip{\e^{s \mathcal L} \mathcal L \psi}{\mathcal L \psi}
    % &\qquad
    % = \ip{\e^{s \mathcal L}p}{p}
    % - \ip{\e^{s \mathcal L} (\id - \Pi_p) \mathcal L \psi}{(\id - \Pi_p) \mathcal L \psi}
    % - \ip{\e^{s \mathcal L} (\id - \Pi_p) \mathcal L \psi}{\Pi_p \mathcal L \psi}
    % - \ip{\e^{s \mathcal L} \Pi_p \mathcal L \psi}{\mathcal L \psi} \\
    % &\qquad
      &= \ip{\e^{s \mathcal L} p}{p + (\id - \Pi_p) \mathcal L \psi}
    - \ip{\e^{s \mathcal L} \bigl(p + (\id - \Pi_p) \mathcal L \psi\bigr)}{(\id - \Pi_p) \mathcal L \psi} \\
    &\qquad \quad - \ip{\e^{s \mathcal L} (\id - \Pi_p) \mathcal L \psi}{\Pi_p \mathcal L \psi}
    - \ip{\e^{s \mathcal L} \Pi_p \mathcal L \psi}{\mathcal L \psi}.
\end{align*}
The last two terms are bounded using the general bound on the Langevin semigroup~\eqref{eq:decay_semigroup_general}:
\[
    \abs{\ip{\e^{s \mathcal L} (\id - \Pi_p) \mathcal L \psi}{\Pi_p \mathcal L \psi}}
    \vee \abs{\ip{\e^{s \mathcal L} \Pi_p \mathcal L \psi}{\mathcal L \psi}}
    \leq M \exp \left( - \lambda \min(\gamma, \gamma^{-1}) s\right) \norm{\Pi_p \mathcal L \psi} \norm{\mathcal L \psi},
\]
and the first two terms are bounded using \cref{proposition:semigroup_meanzero_observable}:
\begin{align*}
    \abs{\ip{\e^{s \mathcal L} p}{p + (\id - \Pi_p) \mathcal L \psi}}
    &\leq C \norm{p + (\id - \Pi_p) \mathcal L \psi} \zeta(t), \\
    \abs{\ip{\e^{s \mathcal L} \bigl(p + (\id - \Pi_p) \mathcal L \psi\bigr)}{(\id - \Pi_p) \mathcal L \psi}}
    &\leq C \norm{(\id - \Pi_p) \mathcal L \psi}[5] \norm{p + (\id - \Pi_p) \mathcal L \psi} \zeta(t),
\end{align*}
where $\zeta(t) = \gamma^{-2} \e^{- \frac{\lambda}{\gamma}s } + \e^{- \lambda \gamma s}$.
For the second inequality,
we used that the right-hand side of~\eqref{eq:optimal_decay_correlation} in \cref{proposition:semigroup_meanzero_observable} is a bound from above also for $\abs{\ip*{\e^{t \mathcal L^*}\!\! f}{h}} = \abs{\ip{f}{\e^{t \mathcal L}h}}$,
because the operator $\mathcal L^*$,
which is the formal $L^2(\mu)$ adjoint of $\mathcal L$,
coincides with $\mathcal L$ up to the sign of the Hamiltonian part.
\end{proof}

The constant on the right-hand side is smaller when $\psi \approx \phi$,
and when $\psi = 0$ we recover the previous bound.


\paragraph{Variance of the estimator.}%
For the variance, we can obtain a crude bound using
\begin{align*}
    \mathrm{Var} (v(t))
    &\leq \expect \left( \abs{v(t) - D_{\psi}}^2 \right)
    = \frac{1}{4 t^2} \expect \big( |q_t - q_0 - \xi_t|^2 |q_t - q_0 + \xi_t|^2 \big) \\
    &\leq \frac{1}{4 t^2} \sqrt{\expect \left(  |q_t - q_0 - \xi_t|^4 \right)} \sqrt{\expect \left( |q_t - q_0 + \xi_t|^4 \right)}.
\end{align*}
We now bound
\begin{align*}
    \expect \left( |q_t - q_0 + \xi_t|^4 \right)
    &= \expect \left( \abs{\phi_t - \phi_0 - \psi_t + \psi_0 + I_{\psi} - I_{\phi}}^4 \right) \\
    &\leq 3^3 \left( \expect \left( \abs{\phi_t - \psi_t}^4 \right) + \expect \left( \abs{\phi_0 - \psi_0}^4 \right) + \expect \left( \abs{I_{\psi} - I_{\phi}}^4 \right) \right).
\end{align*}
The first two terms are bounded by $\norm{\phi - \psi}_{L^4(\mu)}^4$.
Using a moment inequality for It\^o integrals, the last term can be bounded as
\[
    \expect \left( \abs{I_{\psi} - I_{\phi}}^4 \right) \leq 6 t \expect \int_{0}^{t} \abs{\derivative{1}[\phi]{p}(q_s, p_s) - \derivative{1}[\psi]{p}(q_s, p_s)}^4 \, \d s = 6 t \int \abs{\derivative{1}[\phi]{p} - \derivative{1}[\psi]{p}}^4 \d \mu.
\]
Likewise, the other term can be bounded as
\begin{align*}
    \expect \left( |q_t - q_0 - \xi_t|^4 \right)
    &\leq 3^3 \left( \expect \left( 2 \norm{\phi + \psi}[L^4(\mu)]^4 + 6 t \norm{\derivative{1}[\phi]{p} + \derivative{1}[\psi]{p}}[L^{4}(\mu)]^4 \right) \right).
\end{align*}

\section{Toy example: Langevin dynamics in one dimension}%
\label{sec:toy_example_langevin_dynamics_in_one_dimension}



% \paragraph{Estimator diffusion coefficient.}%
% \label{par:estimator_diffusion_coefficient}

% Let $U_t = \sqrt{2 D} W_t$ and consider the estimator
% \[
%     2 \hat D = \frac{t_1 U_{t_1}^2 + \dotsb + t_N U_{t_N}^2}{t_1^2 + \dotsb + t_N^2},
% \]
% where $t_i = i \Delta_t$.
% This estimator is clearly unbiased: $\expect (\hat D) = D$.
% Let $\xi_i = (U_{t_{i}} - U_{t_{i-1}})/\sqrt{2D \Delta t}$, for $i = 1, \dotsc, N$.
% It holds
% \begin{align*}
%     \frac{\hat D - D}{D}
%     &= 6 \left( \frac{ \sum_{i=1}^{N} \sum_{j=i}^{N} (\xi_i ^2 - 1)  j  + 2 \sum_{i=1}^{N} \sum_{j=i+1}^{N} \sum_{k=i}^{j} \xi_i \xi_j k}
%     {1 + \dotsb + N^2} \right) \\
%     &= 3 \left( \frac{ \sum_{i=1}^{N} (\xi_i ^2 - 1)  (N+1-i)(N+i)  + 2 \sum_{i=1}^{N} \sum_{j=i+1}^{N} \xi_i \xi_j (j+1-i)(j+i)}
%     { N(N+1)(2N+1)} \right).
% \end{align*}
% We calculate
% \begin{align*}
%     \expect \abs{ \frac{\hat D - D}{D} }^2
%     &= 9 \left( \frac{ \sum_{i=1}^{N} \expect (\xi_i ^4 + 1 - 2 \xi_i^2)  (N+1-i)^2 (N+i)^2}
%     { N^2 (N+1)^2 (2N+1)^2} \right) \\
%     &\quad + 36 \left(  \frac{\sum_{i=1}^{N} \sum_{j=i+1}^{N} \expect \left( \abs{\xi_i}^2 \abs{\xi_j}^2 \right) \abs{j+1-i}^2 \abs{j+i}^2}
%     { N^2 (N+1)^2 (2N+1)^2} \right) \\
%     &= 18 \left( \frac{ \sum_{i=1}^{N}  (N+1-i)^2 (N+i)^2}
%     { N^2 (N+1)^2 (2N+1)^2} \right)
%     + 36 \left(  \frac{\sum_{i=1}^{N} \sum_{j=i+1}^{N} \abs{j+1-i}^2 \abs{j+i}^2}
%     { N^2 (N+1)^2 (2N+1)^2} \right).
% \end{align*}
\appendix
\section{Proof of \cref{proposition:semigroup_meanzero_observable}}%
\label{sec:auxiliary_technical_results}

The proof is based on several lemmata.
In order to state the results, we define the probability measures

\begin{equation}
    \label{eq:definition_prob_measures}
    \nu(\d q) = \frac{\e^{- \beta V(q)} \, \d q}{\int_{\torus^d}\e^{-\beta V(\widetilde q)} \d \widetilde q},
    \qquad \kappa(\d p) = \left( \frac{\beta}{2 \pi} \right)^{d/2}\exp \left( - \beta \frac{\abs{p}^2}{2} \right) \d p.
\end{equation}
For a measure $\pi$, we define the weighted Sobolev space $H^i(\pi)$ as the subspace of $L^2(\pi)$
of functions whose derivatives up to order $i$ are in $L^2(\pi)$.
The associated norm is given by
\[
    \norm{u}_i^2 = \norm{f}^2 + \norm*{\nabla f}^2 + \dotsb + \norm*{\nabla^i f}^2,
\]
where  $\nabla^j f$ is the tensor containing the $j$-th order derivatives of $f$ and $\norm{\dummy}$ is the norm of~$L^2(\pi)$,
generalized to tensors in the usual manner.
We also define $H^{i}_0(\pi) = H^i(\pi) \cap L^2_0(\pi)$,
and we recall that a probability measure $\pi$ is said to satisfy the Poincaré inequality with constant $R$ if
\begin{equation}
    \label{eq:poincare}
    \tag{P$_R$}
    \forall f \in H^1_0(\pi), \qquad
    \norm*{f}^2 \leq \frac{1}{2R} \norm{\grad f}^2.
\end{equation}

% In the lemmata below, we focus on the periodic, one-dimensional case.
We also define $\nabla_q^* = \beta \nabla V(q) - \nabla_q$ and $\nabla_p^* = \beta p - \nabla_p$,
and note that these operators are formally the $L^2(\mu)$ adjoints of $\nabla_q$ and $\nabla_p$.
Similarly, in one dimension we write $\partial_q^* = \beta V'(q) - \partial_q$ and $\partial_p^* = \beta p - \partial_p$.
The first lemma shows the exponential convergence of the derivatives of the overdamped Langevin semigroup.
\begin{lemma}
    \label{lemma:overdamped_langevin_decay_derivatives}
    Let $\mathcal X = \real^d$ or $\mathcal X = \torus^d$,
    and let $W: \mathcal X \to \real$ be a smooth potential such that the probability measure
    \[
        \pi(\d x) = \frac{\e^{- \beta W(x)} \d x}{\int_{\real^d} \e^{-\beta W(\widetilde x)} \d \widetilde x}
    \]
    satisfies~\eqref{eq:poincare}.
    Let also $\mathcal L_{\rm ovd}^W = - W'(p) \partial_p + \beta^{-1} \partial_p^2$ denote the generator of overdamped Langevin dynamics in potential~$W$.
    If $f \in H^i_0(\pi) \cap C^{\infty}(\mathcal X)$ for some $i \geq 0$,
    then $\e^{t \mathcal L_{\rm ovd}} f \in H^i_0(\pi)$ and there exists $K = K(i)$ such that
    \[
        \norm*{\e^{t \mathcal L_{\rm ovd}^W} f}_i \leq K \e^{- 2 R t} \norm*{f}_i.
    \]
\end{lemma}
\begin{proof}
    For simplicity, we consider only the case where $\mathcal X = \torus^d$,
    so that we know \emph{a priori} that~$\e^{t \mathcal L_{\rm ovd}^W} f \in H^i_0(\torus^d)$ because
    the state space is compact and $\e^{t \mathcal L_{\rm ovd}^W} f \in C^{\infty}(\torus^d)$ by ellipticity.
    To lighten notations,
    we also confine ourselves to the one-dimensional setting $d = 1$,
    but the proof carries over \emph{mutatis mutandis} to the multi-dimensional case.

    We use the notations $u(t) = \e^{t \mathcal L_{\rm ovd}^W} f$ and
     $\seminorm{h}_j = \norm*{(- \mathcal L_{\rm ovd}^W)^{j/2} h}$ for $j \geq 0$.
    Since $\mathcal L_{\rm ovd}$ commutes with~$(-\mathcal L_{\rm ovd}^W)^{i/2}$,
    it holds that
    \[
        \frac{1}{2} \derivative*{1}{t} \seminorm{u}_j^2 = \ip{\mathcal L_{\rm ovd}^W (-\mathcal L_{\rm ovd})^{j/2} u}{(- \mathcal L_{\rm ovd}^W)^{j/2} u}.
    \]
    Introducing $\partial_x^* = \beta W' - \partial_x$
    and noting that $\mathcal L_{\rm ovd}^W = - \partial_x^* \partial_x$,
    we obtain
    \[
        \frac{1}{2} \derivative*{1}{t} \seminorm{u}_j^2 = - \norm*{\partial_x (-\mathcal L_{\rm ovd}^W)^{j/2} u}.
    \]
    Since $(-\mathcal L_{\rm ovd}^W)^{j/2} u \in L^2_0(\pi)$,
    we can apply Poincar\'e's inequality~\eqref{eq:poincare},
    which gives
    \[
        \frac{1}{2} \derivative*{1}{t} \seminorm{u}_j^2 \leq - 2R \seminorm{u}_j^2,
    \]
    implying the exponential convergence estimate
    \begin{equation}
        \label{eq:exponential_convergence}
        \forall t \geq 0, \qquad
        \seminorm{u(t)}_i \leq \e^{- 2 R t} \seminorm{u(0)}_i.
    \end{equation}
    Applying this estimate with $j = 0$ gives the usual convergence estimate for the norm $\norm{u}$.
    For~$j = 1$, it holds for sufficiently regular $h$ that
    \[
        \seminorm{h}_1 = \norm*{(- \mathcal L_{\rm ovd}^W)^{1/2} h} = \sqrt{\ip{\mathcal L_{\rm ovd}^W h}{h}} = \beta^{-1} \norm{\partial_x h},
    \]
    so~\eqref{eq:exponential_convergence} implies the exponential convergence of $\norm{\partial_x u}$.
    For $j = 2$, we calculate using the commutator relation $\commut{\partial_x}{\partial_x^*} = \beta V''$ that
    \[
        \seminorm{h}_2^2 = \norm*{- \mathcal L_{\rm ovd}^Wf}^2
        = \beta^{-2} \ip{\partial_x^* \partial_x h}{\partial_x^* \partial_x h}
        = \beta^{-2} \norm*{\partial_x^2 h}^2 + \beta^{-1} \ip{V'' \partial_x h}{\partial_x h}.
    \]
    Therefore~\eqref{eq:exponential_convergence} implies that
    \begin{align*}
        \norm*{\partial_x^2 u}^2
        &\leq \beta \abs{\ip{V'' \partial_x u}{\partial_x u}} + \e^{-4Rt} \bigl( \norm{\partial_x^2 u(0)}^2 + \beta \ip{V'' \partial_x u(0)}{\partial_x u(0)} \bigr) \\
        &\leq \beta \norm*{V''}_{\infty} \norm{\partial_x u}^2 + \e^{-4Rt} \bigl( \norm{\partial_x^2 u(0)}^2 + \beta \norm*{V''}_{\infty} \norm*{\partial_x u(0)}^2 \bigr),
    \end{align*}
    and using the exponential convergence of the first term on the right-hand side,
    which was proved in the previous step,
    allows to conclude that $\norm*{\partial_x^2 u} \leq K \e^{-2 Rt} \norm*{u}^2$ for some appropriate constant $K \geq 1$.
    This procedure can then be repeated in order to deduce the statement.
\end{proof}

\newcommand{\auxnorm}[1]{|\!|\!| #1 |\!|\!|}
\begin{remark}
    An alternative approach for showing~\cref{lemma:overdamped_langevin_decay_derivatives} is
    to define an auxiliary norm
    \[
        \auxnorm{u}_N^2 = \norm*{u}^2 + a_1 \norm*{\partial_x u}^2 + \dotsc + a_N \norm*{\partial_x^N u}^2,
    \]
    with coefficients $a_i = \varepsilon^i$ for some $\varepsilon \in (0, 1)$.
    This norm is equivalent to the weighted Sobolev norm $\norm*{u}_N$,
    and it is possible to show for all $\lambda \in (0, 1)$ that
    \[
        \frac{1}{2}\derivative*{1}{t} \auxnorm{u}_N^2 \leq - 2 R(1 - \lambda) \, \auxnorm{u}_N^2
    \]
    for $\varepsilon$ sufficiently small.
    The reason for the suboptimal rate here is that the term $\norm{\partial_x u}^2$,
    obtained from $\frac{1}{2} \partial_t \norm*{u}^2$,
    needs to control the two terms $\norm{u}^2 + a_1 \norm{\partial_x u}^2$.
\end{remark}


The next lemma provides an estimate on the convergence of the solution to the backward Kolmogorov equation for Langevin dynamics
in the overdamped limit $\gamma \to \infty$,
in the case of an initial condition depending only on $q$.
We note that the connection between Langevin and overdamped Langevin dynamics has been widely studied in the literature.
See, for example, \cite[Theorem 10.1]{MR0214150} for a proof of the convergence of trajectories uniformly over compact time intervals with probability 1,
and~\cite{MR4054345} for a weak convergence result.
See also~\cite{MR496218,MR918689} for formal asymptotic expansions of the solution to the Fokker--Planck equation in the overdamped limit.

\begin{lemma}
    \label{lemma:backward_kolmogorov_obs_q}
    Let $\mathcal L_{\rm ovd} = - V'(q) \partial_q + \beta^{-1} \partial_q^2$ and $f \in L^2_0(\nu) \cap C^{\infty}(\torus)$,
    where $\nu$ is given in~\eqref{eq:definition_prob_measures}.
    Let also $\bar f(q, p) = f(q)$ and
    \[
        \widehat u(t) = \e^{t\mathcal L_{\rm ovd}} f, \qquad
        u(t) = \e^{t \gamma\mathcal L} \bar f.
    \]
    Then there exist positive constants $C$ and $\lambda$ independent of $f$ and $\gamma$ such that
    \[
        \forall \gamma \geq 1, \qquad
        \forall t \geq 0, \qquad
        \norm{u(t)  - \widehat u(t)} \leq
        C \norm{f}_4 \gamma^{-1} \e^{-\lambda t}.
    \]
\end{lemma}
\begin{proof}
    Throughout this proof, $C$ denotes a positive constant that can change from occurrence to occurrence but is independent of $\gamma$ and $f$.
    We recall that there exists a positive constant $\lambda_{\rm Lang}$ such that~\cite{roussel2018spectral,pavliotis2011applied}
    \begin{equation}
        \label{eq:decay_langevin}
        \forall \gamma > 0, \qquad
        \norm{\e^{t \gamma \mathcal L_{\rm Lang}}}[\mathcal B\left(L^2_0(\mu)\right)] \leq C \e^{- \lambda_{\rm Lang} \min\{1,\gamma^2\} t}.
    \end{equation}
    In addition, \cref{lemma:overdamped_langevin_decay_derivatives} implies the existence of $\lambda_{\rm ovd} > 0$ independent of $i$ such that
    \begin{equation}
        \label{eq:decay_ovd}
        \norm{\e^{t \mathcal L_{\rm ovd}}}[\mathcal B\left(H^i_0(\mu)\right)] \leq C \e^{- \lambda_{\rm ovd} t}.
    \end{equation}
    Based on formal asymptotics expansions similar to those in~\cite[Chapter 6]{pavliotis2011applied},
    we define the function
    \(
        \widetilde u(q, p, t) =
        \widehat u(q, t)
        + \gamma^{-1} p \, \partial_q \widehat u(q, t)
        + \gamma^{-2} (p^2 - \beta^{-1}) \partial_q^{2} \widehat u.
    \)
    An explicit calculation gives
    \begin{align*}
        (\partial_t - \gamma \mathcal L) \widetilde u
        &= \gamma^{-1} p \, \partial_t \partial_q \widehat u(q, t) + \gamma^{-2} (p^2 - \beta^{-1}) \partial_t \partial_q^{2} \widehat u
        - \gamma^{-1} p (p^2 - \beta^{-1}) \partial_q^{3} \widehat u + 2 \gamma^{-1} p \, V'(q) \, \partial_q^2 \widehat u \\
        &= - \gamma^{-1} p \, \partial_q \partial_q^* \partial_q \widehat u(q, t) - \gamma^{-2} (p^2 - \beta^{-1})  \partial_q^{2} \partial_q^* \partial_q \widehat u \\
        &\qquad - \gamma^{-1} p (p^2 - \beta^{-1}) \partial_q^{3} \widehat u + 2 \gamma^{-1} p \, V'(q) \, \partial_q^2 \widehat u.
    \end{align*}
    The right-hand side, which we denote by $\gamma^{-1} r(t)$, has values in $L^2_0(\mu)$.
    By the general Leibniz rule gives,
    it is simple to show that
    \begin{equation}
        \label{eq:commutators_derivatives}
        \commut{\partial_q^N}{\partial_q^*}
        = \commut{\partial_q^N}{\beta V' - \partial_q}
        = \beta \commut{\partial_q^N}{V'}
        = \beta \sum_{k=1}^{N} {N \choose k} V^{(k+1)} \partial_q^{(N-k)},
    \end{equation}
    which can be employed to show that $\norm{r(t)} \leq C \norm{\widehat u(t)}_4$,
    and so $\norm{r(t)} \leq C \e^{-\lambda_{\rm ovd} t} \norm{f}_4$ by~\eqref{eq:decay_ovd}.
    Using the notation $e(t) = u(t) - \widetilde u(t)$,
    we calculate that $e$ satisfies the equation
    \[
        \partial_t e = \gamma \mathcal L e - \gamma^{-1} r, \qquad
        e(0) = \gamma^{-1} p \, f'(q) + \gamma^{-2} (\beta^{-1} - p^2) f''(q).
    \]
    By Duhamel's formula,
    this implies
    \[
        e(t) = \e^{t \gamma \mathcal L} \bigl( e(0) \bigr) + \gamma^{-1} \int_{0}^{t} \e^{(t- s) \gamma \mathcal L} r(s) \, \d s.
    \]
    From~\eqref{eq:decay_langevin} we deduce immediately that
    \begin{align*}
        e(t)
        &\leq C \e^{- \lambda_{\rm Lang} t} \norm{e(0)}
        + C \gamma^{-1} \int_{0}^{t} \e^{- \lambda_{\rm Lang} (t-s)} \e^{-\lambda_{\rm ovd} s} \d s \\
        &\leq C \e^{- \lambda t} \norm{e(0)} + C \norm{f}_4 \gamma^{-1} t \e^{- \lambda t},
    \end{align*}
    for $\lambda = \min(\lambda_{\rm ovd}, \lambda_{\rm Lang})$.
    The result then follows after noticing that $\norm{e(0)} \leq C \gamma^{-1} \norm{f}_2$ and
    \[
        \norm{\widetilde u(t) - \widehat u(t)} \leq C \norm{f}_2 \gamma^{-1} \e^{- \lambda_{\rm ovd} t},
    \]
    by \eqref{eq:decay_ovd}.
\end{proof}

The next lemma provides an intermediate result for proving~\cref{proposition:semigroup_meanzero_observable}.
The result proved here is sharper than what would be obtained from a simple application of~\eqref{eq:decay_langevin},
but not yet sufficient for obtaining optimal estimates for the bias of estimator~\eqref{eq:simple_estimator}.
\begin{lemma}
    \label{lemma:initial_lemma}
    Assume that $f \in H^2(\mu)$ is a smooth function such that
    \begin{equation}
        \label{eq:assumption_f}
        \int f(q, p) \, \kappa(\d p) = 0.
    \end{equation}
    Then there exist constants $C$ and $\lambda$ independent of $\gamma$ and $f$ such that
    \[
        \forall \gamma > 1, \qquad
        \forall t \geq 0, \qquad
        \norm*{\e^{t\mathcal L} f}
        \leq C \norm{f}[2]
        \left( \e^{- \gamma t} + \gamma^{-1} \e^{-\frac{\lambda t}{\gamma}} \right).
    \]
\end{lemma}
\begin{proof}
    We prove the result for functions $f(q, p)$ of the form
    \begin{equation}
        \label{eq:expansion}
        f(q, p) = \sum_{i=0}^{N} \sum_{j=1}^{N} c_{ij} \, G_i(q) H_j(p).
    \end{equation}
    where $G_i = {\rm Re}(\e^{ix})$ and $H_j$ denotes the Hermite polynomial of degree $j$.
    The space of functions of this form is dense in $(\id - \Pi_p) H^2(\mu)$,
    so the general result follows by density.

    We consider the following decomposition of the generator:
    \[
        \mathcal L
        = \left( p \derivative{1}{q} - \derivative*{1}[V]{q}(q) \derivative{1}{p} \right)
        + \gamma \left( - p \derivative{1}{p} + \beta^{-1} \derivative{2}{p^2} \right)
        =: \mathcal L_{\rm Ham} + \gamma \mathcal L_{\rm FD}.
    \]
    Let $v(t) = \e^{t \mathcal L} f(q, p) - \e^{- t \gamma \mathcal L_{\rm FD}} f(q, p)$.
    In the expression $\e^{- t \gamma \mathcal L_{\rm FD}} f(q, p)$,
    the variable $q$ should should be viewed as a parameter.
    The function $v$ satisfies the initial value problem
    \[
        \partial_t v = \mathcal L v +  \mathcal L_{\rm Ham} \bigl(\e^{t \gamma \mathcal L_{\rm FD}} f\bigr), \qquad v(0) = 0.
    \]
    Using Duhamel's formula, we have
    \[
        v(t) = \int_{0}^{t} \e^{- (t-s) \mathcal L}  \Bigl( \mathcal L_{\rm Ham} \bigl(\e^{s \gamma \mathcal L_{\rm FD}} f\bigr) \Bigr) \, \d s,
    \]
    and therefore
    \begin{equation}
        \label{eq:intermediate_decay_correlation}
        \e^{t \mathcal L} f =  \e^{t \gamma \mathcal L_{\rm FD}} f
        + \int_{0}^{t} \e^{- (t-s) \mathcal L}  \Bigl( \mathcal L_{\rm Ham} \bigl(\e^{s \gamma \mathcal L_{\rm FD}} f\bigr) \Bigr) \, \d s.
    \end{equation}
    By~\eqref{eq:assumption_f} and with the notation $\norm{\dummy}[\kappa]$ for the norm of $L^2(\kappa)$,
    the first term is bounded as
    \begin{align}
        \notag
        \norm{\e^{t \gamma \mathcal L_{\rm FD}} f(q, p)}^2
        &= \int \!\!\!\! \int  \abs{\e^{t \gamma \mathcal L_{\rm FD}} f(q, p) }^2 \d \kappa(p) \,\d \nu(q)
        = \int \norm{\e^{t \gamma \mathcal L_{\rm FD}} f(q, \cdot) }[\kappa]^2 \, \d \nu(q) \\
        \label{eq:bound_first_term}
        &\leq \int \e^{-\gamma t} \norm{f(q, \cdot) }[\kappa]^2 \, \d \nu(q) = \e^{-\gamma t} \norm{f}^2.
    \end{align}
    For the second term, since $\commut{\partial_p}{\partial_p^*} = \beta$ and $\commut{\partial_q}{\partial_q^*} = \beta V''$,
    we have
    \begin{align*}
        \norm{\partial_q \partial_p^* \e^{t \gamma \mathcal L_{\rm FD}} f}
        &= \norm{\partial_q \partial_p \e^{t \gamma \mathcal L_{\rm FD}} f} + \beta \norm{\partial_q \e^{t \gamma \mathcal L_{\rm FD}} f}, \\
        \norm{\partial_q^* \partial_p \e^{t \gamma \mathcal L_{\rm FD}} f}
        &= \norm{\partial_q \partial_p \e^{t \gamma \mathcal L_{\rm FD}} f}
        + \beta \ip{V''(q) \partial_p \e^{t \gamma \mathcal L_{\rm FD}} f}{\partial_p \e^{t \gamma \mathcal L_{\rm FD}} f} \\
        &\leq \norm{\partial_q \partial_p \e^{t \gamma \mathcal L_{\rm FD}} f}
        + \beta \norm{V''}[\infty] \norm{\partial_p \e^{t \gamma \mathcal L_{\rm FD}} f}.
    \end{align*}
    Since $f$ is assumed to be a finite linear combination of the form~\eqref{eq:expansion} and
    Hermite polynomials are the eigenfunctions of $\mathcal L_{\rm FD}$,
    we can freely change the order of the operators $\partial_q$, $\partial_p$ and $\e^{t \gamma \mathcal L_{\rm FD}}$.
    From these equations we deduce
    \begin{align}
        \notag
        \norm{\mathcal L_{\rm Ham} \e^{t \gamma \mathcal L_{\rm FD}} f}
        &= \beta^{-1} (\partial_q \partial_p^* - \partial_q^* \partial_p) \e^{t \gamma \mathcal L_{\rm FD}} f \\
        \label{eq:reasoning_action_lham}
        &\leq 2 \beta^{-1} \norm{\partial_q \partial_p \e^{t \gamma \mathcal L_{\rm FD}} f}
        + \norm{\partial_q \e^{t \gamma \mathcal L_{\rm FD}} f}
        + \norm*{V''}[\infty] \norm{\partial_p \e^{t \gamma \mathcal L_{\rm FD}} f}.
    \end{align}
    % Since $\mathcal L_{\rm FD}$ is not hypoelliptic when viewed as an operator acting on functions of $q$ and $p$,
    % it is not clear \emph{a priori} that~$\e^{t \gamma \mathcal L_{\rm FD}} f$ is a smooth function.
    From \cref{lemma:overdamped_langevin_decay_derivatives} and the fact that $\kappa$ satisfies~\eqref{eq:poincare} with constant $R = \frac{1}{2}$,
    we immediately obtain
    \begin{align*}
        \norm{\mathcal L_{\rm Ham} \e^{t \gamma \mathcal L_{\rm FD}} f}
        &\leq C \e^{-\gamma t} \bigl( \norm{f} + \norm{\partial_q f} + \norm{\partial_p \partial_q f}\bigr).
    \end{align*}
    Going back to~\eqref{eq:intermediate_decay_correlation}, using~\eqref{eq:decay_langevin},
    and denoting $\widetilde \lambda_{\rm Lang} = \min \{\frac{1}{2}, \lambda_{\rm Lang} \}$,
    we obtain
    \begin{align*}
        \norm*{ \e^{t \mathcal L} f}
        &\leq  \e^{-\gamma t} \norm{f}
        + C  \bigl( \norm{f} + \norm{\partial_q f} + \norm{\partial_p \partial_q f}\bigr) \int_{0}^{t} \e^{-\frac{\lambda_{\rm Lang}}{\gamma}(t-s)}  \, \e^{-\gamma s} \, \d s \\
        % &\leq \norm{Q} \norm{P} \e^{-\gamma t}
        % + C \norm{Q}_1 \norm{P}_1 \int_{0}^{t} \e^{-\frac{\min \left\{ \lambda_{\rm Lang}, \frac{1}{2} \right\} }{\gamma}(t-s)}  \, \e^{-\gamma s} \, \d s \\
        &\leq  \e^{-\gamma t} \norm{f}
        + C \bigl( \norm{f} + \norm{\partial_q f} + \norm{\partial_p \partial_q f}\bigr)
        \left( \frac{\e^{- \frac{\widetilde \lambda_{\rm Lang}}{\gamma}t} - \e^{- \gamma t}}{\gamma - \frac{\widetilde \lambda_{\rm Lang}}{\gamma}} \right) \qquad \forall \gamma \geq 1,
    \end{align*}
    which allows to conclude.
\end{proof}

% \begin{lemma}
%     \label{lemma:initial_lemma}
%     Assume that $f(q, p) = Q(q) P(p)$ for smooth functions $Q \in H^1(\nu)$ and $P \in H^1_0(\kappa)$.
%     Then there exist constants $C$ and $\lambda$ independent of $\gamma$, $Q$ and $P$ such that
%     \[
%         \forall \gamma > 1, \qquad
%         \forall t \geq 0, \qquad
%         \norm*{\e^{t\mathcal L} f}
%         \leq C \norm*{Q}_1 \norm*{P}_1
%         \left( \e^{- \gamma t} + \gamma^{-1} \e^{-\frac{\lambda t}{\gamma}} \right).
%     \]
% \end{lemma}
% \begin{proof}
%     We consider the following decomposition of the generator:
%     \[
%         \mathcal L
%         = \left( p \derivative{1}{q} - \derivative*{1}[V]{q}(q) \derivative{1}{p} \right)
%         + \gamma \left( - p \derivative{1}{p} + \beta^{-1} \derivative{2}{p^2} \right)
%         =: \mathcal L_{\rm Ham} + \gamma \mathcal L_{\rm FD}.
%     \]
%     Let $v(t) = \e^{t \mathcal L} f(q, p) - Q(q) \e^{- t \gamma \mathcal L_{\rm FD}} P(p)$.
%     The function $v$ satisfies the initial value problem
%     \[
%         \partial_t v = \mathcal L v +  \mathcal L_{\rm Ham} \bigl(Q(q) \e^{t \gamma \mathcal L_{\rm FD}} P(p)\bigr), \qquad v(0) = 0.
%     \]
%     Using Duhamel's formula, we have
%     \[
%         v(t) = \int_{0}^{t} \e^{- (t-s) \mathcal L}  \Bigl( \mathcal L_{\rm Ham} \bigl(Q(q) \e^{s \gamma \mathcal L_{\rm FD}} P(p)\bigr) \Bigr) \, \d s,
%     \]
%     and therefore
%     \begin{equation}
%         \label{eq:intermediate_decay_correlation}
%         \e^{t \mathcal L} f(q,p) = Q(q) \e^{t \gamma \mathcal L_{\rm FD}} P(p)
%         + \int_{0}^{t} \e^{- (t-s) \mathcal L}  \Bigl( \mathcal L_{\rm Ham} \bigl(Q(q) \e^{s \gamma \mathcal L_{\rm FD}} P(p)\bigr) \Bigr) \, \d s.
%     \end{equation}
%     We calculate
%     \begin{align*}
%         \mathcal L_{\rm Ham} \bigl(Q(q) \e^{t \gamma \mathcal L_{\rm FD}} P(p)\bigr)
%         &= Q'(q) p \e^{t \gamma \mathcal L_{\rm FD}} P(p) - V'(q) Q(q) \partial_p(\e^{t \gamma \mathcal L_{\rm FD}} P) (p) \\
%         &=: Q_1(q) P^t_1(p) + Q_2(q) P^t_2(p).
%     \end{align*}
%     From \cref{lemma:overdamped_langevin_decay_derivatives} and the fact that $\kappa$ satisfies~\eqref{eq:poincare} with constant $R = \frac{1}{2}$,
%     we immediately obtain the bound
%     \(
%         \norm{P^t_2} \leq \e^{-2\gamma t} \norm*{P'}.
%     \)
%     Likewise,
%         % &= \ip{\partial_p (\e^{t \gamma \mathcal L_{\rm FD}} P)}{\partial_p (\e^{t \gamma \mathcal L_{\rm FD}} P)}
%         % = - \ip{\partial_p^* \partial_p (\e^{t \gamma \mathcal L_{\rm FD}} P)}{\e^{t \gamma \mathcal L_{\rm FD}} P} \\
%         % &= - \ip{\mathcal L_{\rm FD} (\e^{t \gamma \mathcal L_{\rm FD}} P)}{\e^{t \gamma \mathcal L_{\rm FD}} P}
%         % = - \ip{\mathcal L_{\rm FD}^{1/2} (\e^{t \gamma \mathcal L_{\rm FD}} P)}{\mathcal L_{\rm FD}^{1/2} (\e^{t \gamma \mathcal L_{\rm FD}} P)} \\
%         % &= - \norm*{\e^{t \gamma \mathcal L_{\rm FD}} (\mathcal L_{\rm FD}^{1/2}P)}^2 \leq \e^{-2\gamma t} \norm*{\mathcal L_{\rm FD}^{1/2} P}^2
%         % = \e^{-2\gamma t} \norm*{P'}^2,
%     \begin{align*}
%         \norm{\partial_p^* \e^{t \gamma \mathcal L_{\rm FD}} P}^2
%         &= \ip{\partial_p \partial_p^*\e^{t \gamma \mathcal L_{\rm FD}} P}{\e^{t \gamma \mathcal L_{\rm FD}} P}
%         = \ip{(\beta  + \partial_{p}^* \partial_p) \e^{t \gamma \mathcal L_{\rm FD}} P}{\e^{t \gamma \mathcal L_{\rm FD}} P} \\
%         &= \beta \norm*{\e^{t \gamma \mathcal L_{\rm FD}} P}^2 + \norm{\partial_p \mathcal L_{\rm FD} \e^{t \gamma \mathcal L_{\rm FD}} P}^2
%         \leq \e^{-2\gamma t} \left( \beta \norm*{P}^2 + \norm*{P'}^2 \right),
%     \end{align*}
%     so $\norm{P^t_1} = \beta^{-1} \norm{ (\partial_p^* - \partial_p) \e^{t \gamma \mathcal L_{\rm FD}} P(p)} \leq \e^{- \gamma t} (\beta^{-1/2}\norm{P} +  2\beta^{-1}\norm{P'})$.
%     In addition, it holds that $\norm{Q_1} + \norm{Q_2} \leq C \norm{Q}_1$,
%     so we deduce
%     \[
%         \norm{\mathcal L_{\rm Ham} \bigl(Q(q) \e^{t \gamma \mathcal L_{\rm FD}} P(p)\bigr)}
%         \leq C \e^{-\gamma t} \norm{Q}_1 \norm{P}_1.
%     \]
%     Going back to~\eqref{eq:intermediate_decay_correlation} and using~\eqref{eq:decay_langevin},
%     we obtain
%     \begin{align*}
%         \norm*{ \e^{t \mathcal L} f(q,p) }
%         &\leq \norm{Q} \norm{P} \e^{-\gamma t}
%         + C \norm{Q}_1 \norm{P}_1 \int_{0}^{t} \e^{-\frac{\lambda_{\rm Lang}}{\gamma}(t-s)}  \, \e^{-\gamma s} \, \d s \\
%         % &\leq \norm{Q} \norm{P} \e^{-\gamma t}
%         % + C \norm{Q}_1 \norm{P}_1 \int_{0}^{t} \e^{-\frac{\min \left\{ \lambda_{\rm Lang}, \frac{1}{2} \right\} }{\gamma}(t-s)}  \, \e^{-\gamma s} \, \d s \\
%         &\leq \norm{Q} \norm{P} \e^{-\gamma t}
%         + C \norm{Q}_1 \norm{P}_1
%         \left( \frac{\e^{- \frac{\min\left\{ \lambda_{\rm Lang}, \frac{1}{2} \right\}}{\gamma}t} - \e^{- \gamma t}}{\gamma - \frac{\min\left\{ \lambda_{\rm Lang}, \frac{1}{2} \right\}}{\gamma}} \right) \qquad \forall \gamma \geq 1,
%     \end{align*}
%     which allows to conclude.
% \end{proof}

We are now ready to prove~\cref{proposition:semigroup_meanzero_observable}.
\begin{proof}
    [Proof of \cref{proposition:semigroup_meanzero_observable}]
    We again show the result for functions of the form~\eqref{eq:expansion},
    noting that the general result follows by density of functions of this type in $(\id - \Pi_p) H^5(\mu)$.
    We consider the decomposition
    \begin{align}
        \label{eq:decomposition}
        \Bigl( \mathcal L_{\rm Ham} \bigl(\e^{s \gamma \mathcal L_{\rm FD}} f\bigr) \Bigr)
         =& \Pi_p \Bigl( \mathcal L_{\rm Ham} \bigl(\e^{s \gamma \mathcal L_{\rm FD}} f\bigr) \Bigr)
         + (\id - \Pi_p) \Bigl( \mathcal L_{\rm Ham} \bigl(\e^{s \gamma \mathcal L_{\rm FD}} f\bigr) \Bigr),
    \end{align}
    where $\Pi_p h(q, p) = \int h(q, p) \, \d \kappa(p)$.
    Using this decomposition in~\eqref{eq:intermediate_decay_correlation},
    we obtain
    \begin{align}
        \notag
        \ip{\e^{t \mathcal L} f}{f}
        &= \ip{\e^{t \gamma \mathcal L_{\rm FD}} f}{f}
        + \int_{0}^t \ip{\e^{(t-s) \mathcal L} \Pi_p \Bigl( \mathcal L_{\rm Ham} \bigl(\e^{s \gamma \mathcal L_{\rm FD}} f\bigr) \Bigr)}{f} \, \d s\\
        \label{eq:intermediate_decay_correlation_2}
        &\quad + \int_{0}^t \ip{\e^{(t-s) \mathcal L} (\id - \Pi_p) \Bigl( \mathcal L_{\rm Ham} \bigl(\e^{s \gamma \mathcal L_{\rm FD}} f\bigr) \Bigr)}{f} \, \d s.
    \end{align}
    The first term is bounded as in~\eqref{eq:bound_first_term}.
    In order to bound the second term,
    we use \cref{lemma:backward_kolmogorov_obs_q},
    which gives
    \begin{equation}
        \label{eq:semigroup_on_q_part}
        \e^{u \mathcal L} \Pi_p \Bigl( \mathcal L_{\rm Ham} \bigl(\e^{s \gamma \mathcal L_{\rm FD}} f\bigr) \Bigr)
        =  \e^{(u/\gamma) \mathcal L_{\rm ovd}} \Pi_p \Bigl( \mathcal L_{\rm Ham} \bigl(\e^{s \gamma \mathcal L_{\rm FD}} f\bigr) \Bigr) + R_s(u),
    \end{equation}
    with a remainder term satisfying
    \begin{equation}
        \label{eq:bound_remainder}
        \norm{R_s(u)} \leq C \norm{\Pi_p \Bigl( \mathcal L_{\rm Ham} \bigl(\e^{s \gamma \mathcal L_{\rm FD}} f\bigr) \Bigr)}[4] \gamma^{-1} \e^{-\lambda \gamma^{-1} u}.
    \end{equation}
    Since $\Pi_p \mathcal L_{\rm Ham} = \Pi_p \partial_q^*\partial_p$ and $\e^{s \gamma \mathcal L_{\rm FD}} p = \e^{-\gamma t}p$,
    we have
    \begin{align*}
        \pi(q) := \Pi_p \Bigl( \mathcal L_{\rm Ham} \bigl(\e^{s \gamma \mathcal L_{\rm FD}} f\bigr) \Bigr)
        &= \int \partial_q^* \partial_p \e^{s \gamma \mathcal L_{\rm FD}} f \, \d \kappa(p)
        = \int \partial_q^* \e^{s \gamma \mathcal L_{\rm FD}} f \left(\partial_p^* 1 \right) \, \d \kappa(p) \\
        &= \partial_q^* \ip{f(q, p)}{\beta p}[\kappa] \e^{-\gamma t},
    \end{align*}
    where $\ip{\dummy}{\dummy}[\kappa]$ denotes the inner product of $L^2(\kappa)$.
    Given the assumption~\eqref{eq:expansion},
    it is clear that the right-hand side is infinitely differentiable.
    In addition, since $\norm{p} = \beta^{-1}$, we have
    \begin{align*}
        \norm*{\partial_q^i \pi}[\nu]^2
        = \beta \e^{-2 \gamma t} \int \left| \int \partial_q^i \partial_q^* f(q, p) \, p \, \d \kappa(p) \right|^2 \d \nu(q)
        \leq \e^{-2 \gamma t} \norm{\partial_q^i \partial_q^* f(q, p)}^2.
    \end{align*}
    Using the commutator relations~\eqref{eq:commutators_derivatives} and the fact that $V(q)$ is smooth and periodic,
    we obtain the bound
    \(
        \norm{\pi}[4] \leq \e^{-\gamma t} \sum_{i=1}^{5} \norm{\partial_q^i f} \leq \e^{-\gamma t} \norm{f}[5].
    \)
    We emphasize that the latter bound is rather crude,
    in that the right-hand side includes derivatives with respect to both $p$ and $q$,
    but we employ it for simplicity.
    Going back to~\eqref{eq:bound_remainder}, we deduce
    \[
        \norm{R_s(u)} \leq \norm{f}[5] \,\gamma^{-1} \e^{-\lambda \gamma^{-1} u - \gamma s}.
    \]

    In order to bound the third and final term in~\eqref{eq:intermediate_decay_correlation_2},
    we use \cref{lemma:initial_lemma} which,
    combined to manipulations similar to those in~\eqref{eq:reasoning_action_lham},
    leads to
    \begin{align*}
        % \norm{\e^{u \mathcal L} \Bigl( Q_i(q) \, \bigl(P^t_i(p) - \expect_\kappa P^t_i\bigr) \Bigr)}
        \norm{\e^{u \mathcal L} (\id - \Pi_p) \Bigl( \mathcal L_{\rm Ham} \bigl(\e^{s \gamma \mathcal L_{\rm FD}} f\bigr) \Bigr)}
        &\leq C \norm{ \mathcal L_{\rm Ham} \bigl(\e^{s \gamma \mathcal L_{\rm FD}} f\bigr) }[2]  \left(\e^{-\gamma u} + \gamma^{-1} \e^{-\frac{\lambda u}{\gamma}} \right) \\
        &\leq C \norm{f}[4] \e^{-\gamma s}\left(\e^{-\gamma u} + \gamma^{-1} \e^{-\frac{\lambda u}{\gamma}} \right).
    \end{align*}

    Collecting these bounds in~\eqref{eq:intermediate_decay_correlation_2},
    and noting that the contribution of the first term in~\eqref{eq:semigroup_on_q_part} cancels out because $P$ is mean-zero with respect to $\kappa$,
    we deduce
    \begin{align*}
        \label{eq:intermediate_decay_correlation_3}
        \abs{ \ip{\e^{t \mathcal L} f}{h}}
        % &= \ip{Q(q) \e^{t \gamma \mathcal L_{\rm FD}} P(p)}{f}
        % + \int_{0}^{t} \ip{\e^{- (t-s) \mathcal L}  \Bigl( Q_1(q) P^s_1(p) + Q_2(q) P^s_2(p)\Bigr)}{f} \, \d s \\
        &\leq C \e^{-\gamma t} \norm{f} \norm{h}
        + \int_{0}^t \ip{R_s(t-s)}{f} \, \d s \\
        &\quad + \int_{0}^t C \norm{f}[4] \e^{-\gamma s}\left(\e^{-\gamma (t-s)} + \gamma^{-1} \e^{-\frac{\lambda(t-s)}{\gamma}} \right) \norm{f} \, \d s \\
        &\leq C \e^{-\gamma t} \norm{f} \norm{h}
        + C \gamma^{-1} \, \norm{f}_5  \norm{h} \int_{0}^{t}  \e^{-\lambda \gamma^{-1} (t-s) - \gamma s} \, \d s \\
        &\quad+ C \norm{f}[4] \norm{h} \int_{0}^t \e^{-\gamma s}\left(\e^{-\gamma (t-s)} + \gamma^{-1} \e^{-\frac{(t-s)}{\gamma}} \right) \, \d s.
    \end{align*}
    Calculating the integrals, this finally leads to
    \begin{align*}
        \abs{ \ip{\e^{t \mathcal L} f}{h}}
        \leq C \, \norm{f}_5
        \left( \e^{-\gamma t} +  \gamma^{-2} \e^{-\min\left\{\lambda, \frac{1}{2}\right\} \frac{t}{\gamma}} + t\e^{-\gamma t} \right),
    \end{align*}
    which allows to conclude.
\end{proof}


% \begin{proof}
%     Let $u(t) = \e^{t \mathcal L} p$ and $v(t) = u(t) - p \e^{-\gamma t}$.
%     The function $v$ satisfies the equation
%     \[
%         \partial_t v = \mathcal L v + V'(q) \, \e^{-\gamma t}, \qquad v(0) = 0.
%     \]
%     Using Duhamel's formula, we have
%     \[
%         v(t) = \int_{0}^{t} \e^{- (t-s) \mathcal L} \bigl(V'(q)\bigr) \e^{-\gamma s} \, \d s
%     \]
%     By \cref{lemma:backward_kolmogorov_obs_q},
%     we have
%     \[
%         \e^{- (t-s) \mathcal L} \bigl(V'(q)\bigr)
%         = \e^{- (t-s) \frac{1}{\gamma}\mathcal L_{\rm ov}} V' + R(t),
%         % \leq C \e^{- \frac{\lambda}{\gamma} (t-s)} + \frac{C}{\gamma},
%     \]
%     where the remainder term is bounded as $\norm{R(t)} \leq \frac{C}{\gamma}$.
%     Taking the inner product with $p$,
%     and noting that the constant-in-$p$ part of the integrand vanishes,
%     we obtain
%     \[
%         \ip{v(t)}{p} = \int_{0}^{t} R(s) \e^{-\gamma s} \, \d s
%         \leq \frac{C}{\gamma^2}.
%     \]
%     so we deduce
%     \[
%         \norm{v(t)} \leq \int_{0}^{t} \e^{- \frac{\lambda}{\gamma}(t-s)} \e^{- \gamma s} \, \d s + \frac{C}{\gamma^2}.
%     \]
% \end{proof}




% \bibliographystyle{abbrv}
% \bibliography{main}
% \bibliographystyle{trad-abbrv}
\printbibliography
\end{document}
