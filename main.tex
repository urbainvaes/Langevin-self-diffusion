\documentclass[11pt,a4paper]{article}
\usepackage[utf8]{inputenc}
\usepackage[margin=1in]{geometry}
\usepackage{color}
\usepackage{graphicx}
\usepackage{array}
\usepackage{verbatim}
\usepackage{caption}
\usepackage{subcaption}
\usepackage{amsmath,amsthm,amsfonts,amssymb,latexsym}
\usepackage{bbm}
\usepackage{setspace}
\usepackage{xparse}
\usepackage{epstopdf}
\usepackage{pgf}
\usepackage[colorlinks=true,citecolor=blue]{hyperref}
\usepackage[nameinlink,capitalise]{cleveref}

\usepackage{tikz}
\usepackage{tikz-cd}
\usepackage{pgfplotstable}
\pgfplotsset{compat=1.14}
\usetikzlibrary{patterns}
\usetikzlibrary{calc}
\usetikzlibrary{angles}
\usetikzlibrary{quotes}
\usetikzlibrary{external}

\onehalfspacing
% \setlength{\parskip}{6pt}

\DeclareDocumentCommand\abs{s m} {\IfBooleanTF{#1}{\left|#2\right|}{\left|#2\right|}}
\DeclareDocumentCommand\cont{o m o} {C\IfNoValueF{#1}{^{#1}}(#2\IfNoValueF{#3}{;#3})}
\DeclareDocumentCommand\contc{o m o} {C_c\IfNoValueF{#1}{^{#1}}(#2\IfNoValueF{#3}{;#3})}
\DeclareDocumentCommand\sobolev{m m o} {H^{#1}(#2 \IfNoValueF{#3}{,#3})}
\DeclareDocumentCommand\lp{m m o} {L^{#1}\left(#2 \IfNoValueF{#3}{,#3}\right)}
\DeclareDocumentCommand\norm{s m o} {\IfBooleanTF{#1}{\|#2\|}{\left\|#2\right\|}\IfNoValueF{#3}{_{#3}}}
\DeclareDocumentCommand\seminorm{m o o} {\left|#1\right|\IfNoValueF{#2}{_{#2 \IfNoValueF{#3}{,#3}}}}
\DeclareDocumentCommand\ip{s m m o} {\IfBooleanTF{#1}{\langle #2,#3 \rangle}{\left\langle #2,#3 \right\rangle}\IfNoValueF{#4}{_{#4}}}
\DeclareDocumentCommand\dup{m m o} {\left\langle{#1,#2}\right\rangle\IfNoValueF{#3}{_{#3', #3}}}
\DeclareDocumentCommand\gaussian{O{0} O{I}} {g_{#1, #2}}
\DeclareDocumentCommand\littleo{s o m} {o\IfNoValueF{#2}{_{#2}}\IfBooleanTF{#1}{(#3)}{\left(#3\right)}}
\DeclareDocumentCommand\bigo{s o m} {\mathcal O\IfNoValueF{#2}{_{#2}}\IfBooleanTF{#1}{(#3)}{\left(#3\right)}}


\DeclareMathOperator*{\argmax}{arg\,max}
\DeclareMathOperator*{\argmin}{arg\,min}
\DeclareMathOperator*{\re}{Re}
\DeclareMathOperator*{\trace}{tr}
\DeclareMathOperator{\Span}{span}
\DeclareMathOperator{\sym}{sym}
\DeclareMathOperator{\sign}{sign}
\DeclareMathOperator{\diag}{diag}
\DeclareMathOperator{\e}{e}
\DeclareMathOperator{\id}{id}
\DeclareMathOperator{\offdiag}{offdiag}

\newcommand{\revision}[1]{\textcolor{blue}{#1}}
\renewcommand{\revision}[1]{#1}
\newcommand{\gab}[1]{\textcolor{darkgreen}{#1}}
\newcommand{\commut}[2]{[#1, #2]}
\newcommand{\correlation}[1]{\left< #1 \right>}
\newcommand{\dummy}{\,\cdot\,}
\newcommand{\expect}[0]{\mathbf{E}}
\newcommand{\var}[0]{\mathbf{V}}
\newcommand{\iip}[2]{\left(\!\left(#1, #2\right)\!\right)}
\newcommand{\nat}{\mathbf N}
\newcommand{\poly}{\mathbf P}
\newcommand{\real}{\mathbf R}
\newcommand{\integer}{\mathbf Z}
\newcommand{\torus}{\mathbf T}
\newcommand{\grad}{\boldsymbol \nabla}
\newcommand{\hess}{\nabla^2}
\newcommand{\vect}[1]{\boldsymbol{\mathbf #1}}
\newcommand{\mat}[1]{\vect #1}
\renewcommand{\det}[1]{\mathrm{det} \left( #1 \right)}
\renewcommand{\d}{\mathrm d}
\renewcommand{\t}{\mathsf T}
% \renewcommand{\t}{t}

\makeatletter
\DeclareDocumentCommand \derivative{s m o m}{%
    \def\@der{\IfBooleanTF{#1}{\mathrm{d}}{\partial}}
    \def\@default{%
        \mathchoice{%
                \frac{%
                    \@der\ifnum\pdfstrcmp{#2}{1}=0\else^{#2}\fi {\IfNoValueTF{#3}{}{#3}}
                }{%
                    \@for\@token:={#4}\do{\@der \@token}
                }
            } {%
                \@for\@token:={#4}\do{\@der_\@token} \IfNoValueTF{#3}{}{#3}
            } {} {}
    }
    \IfBooleanTF{#1}{\IfNoValueTF{#3}{\@default}{%
                #3%
                \ifnum\pdfstrcmp{#2}{1}=0'\else%
                \ifnum\pdfstrcmp{#2}{2}=0''\else%
                \ifnum\pdfstrcmp{#2}{3}=0^{(3)}\else%
                \ifnum\pdfstrcmp{#2}{4}=0^{(4)}\else%
                \ifnum\pdfstrcmp{#2}{5}=0^{(5)}\else%
                ^{(#2)}\fi\fi\fi\fi\fi
            }
        }{\@default}
}
\makeatother

\definecolor{darkred}{rgb}{.5,0,0}
\definecolor{darkgreen}{rgb}{0,.5,0}
\definecolor{darkblue}{rgb}{0,0,.5}
\newcommand{\red}[1]{\textcolor{darkred}{#1}}
\newcommand{\green}[1]{\textcolor{darkgreen}{#1}}

\theoremstyle{plain}
\newtheorem{assumption}{Assumption}[section]
\newtheorem{lemma}{Lemma}[section]
\newtheorem{corollary}{Corollary}[section]
\newtheorem{theorem}{Theorem}[section]
\newtheorem{proposition}{Proposition}[section]
\newtheorem{result}{Result}[section]
\newtheorem{remark}{Remark}[section]
\numberwithin{equation}{section}

\newcounter{urbainCounter}
\newcommand{\urbain}[1]{\stepcounter{urbainCounter}\red{\arabic{urbainCounter}.} \green{#1}}
\crefname{equation}{}{}
\crefname{paragraph}{\S\!}{\S}
% \crefname{figure}{Figure}{Figures}
% \crefname{section}{Section}{Sections}

\newcommand{\email}[1]{\href{#1}{#1}}
\newcommand{\orcidcolor}{ORC\textcolor{orcidlogocol}{ID}}
\newcommand{\orcid}[1]{\href{https://orcid.org/#1}{\includegraphics[width=.4cm]{z_orcid.pdf}}}

%---------------- GABRIEL ------------
\usepackage{enumerate}
\newcommand{\eps}{\varepsilon}
\newcommand{\dps}{\displaystyle}
\newcommand{\cX}{\mathcal{X}}
\newcommand{\ri}{\mathrm{i}}
\renewcommand{\leq}{\leqslant}
\renewcommand{\geq}{\geqslant}
\renewcommand{\le}{\leqslant}
\renewcommand{\ge}{\geqslant}
% \usepackage{todonotes}
\usepackage{mathrsfs}

% BODY {{{1
\date{\today}
\title{Langevin dynamics in the underdamped regime: effective diffusion and variance reduction }
\author{%
  % G.A. Pavliotis\thanks{Department of Mathematics, Imperial College London (\email{g.pavliotis@imperial.ac.uk})}%
  % \hspace{2mm}\orcid{0000-0002-3468-9227}%
  % \and G. Stoltz\thanks{CERMICS, \'Ecole des Ponts, France \& MATHERIALS, Inria Paris (\email{gabriel.stoltz@enpc.fr})}
  % \hspace{2mm}\orcid{0000-0002-2797-5938}%
  % \and U. Vaes\thanks{Department of Mathematics, Imperial College London (until October 2020) and MATHERIALS, Inria Paris (since November 2020) (\email{urbain.vaes@inria.fr})}%
  % \hspace{2mm}\orcid{0000-0002-7629-7184}%
}

\begin{document}
\maketitle
Let us first consider the Langevin dynamics in one dimension:
\begin{align*}
    \d q &= p \, \d t, \\
    \d p &= - \derivative*{1}[V]{q}(q) \, \d t - \gamma p \, \d t + \sqrt{2 \gamma \beta^{-1}} \d W_t.
\end{align*}
The generator of the associated Markov semigroup is given by
\[
    \mathcal L = p \derivative{1}{q} - \derivative*{1}[V]{q}(q) \derivative{1}{p} + \gamma \left( - p \derivative{1}{p} + \beta^{-1} \derivative{2}{p^2} \right).
\]
Let $\phi$ denote the solution to
\[
    - \mathcal L \phi = p,
\]
and suppose that $\psi$ denote an approximation of $\phi$.
By It\^o's formula, it holds
\begin{align*}
    \phi(q_t, p_t) - \phi(q_0, p_0) &= - \int_{0}^{t} p_s \, \d s + \sqrt{2 \gamma \beta^{-1}} \int_{0}^{t} \derivative{1}[\phi]{p}(q_s, p_s) \, \d W_s \\
    \psi(q_t, p_t) - \psi(q_0, p_0) &= \int_{0}^{t} (\mathcal L \psi) (q_s, p_s) \, \d s + \sqrt{2 \gamma \beta^{-1}} \int_{0}^{t} \derivative{1}[\psi]{p}(q_s, p_s) \, \d W_s.
\end{align*}
Let also $u(t)$ and $v(t)$ denote the random variables
\[
    u(t) = \frac{\abs{q_t - q_0}^2}{2t}
\]
and
\begin{align*}
    v(t) &= D_{\psi}+ \frac{1}{2t} \left( \abs{q_t - q_0}^2 - \abs{\xi_t}^2\right)
\end{align*}
where $D_{\psi} := \gamma \beta^{-1} \int_{\torus \times \real} \abs{\derivative{1}[\psi]{p}}^2 \, \d \mu$ and
 \[
    \xi_t = \psi(q_t, p_t) - \psi(q_0, p_0) - \sqrt{2 \gamma \beta^{-1}} \int_{0}^{t} \derivative{1}[\psi]{p}(q_s, p_s) \, \d W_s\right).
 \]
It holds
\[
    \lim_{t \to \infty} \expect \bigl( u(t) \bigr) = \lim_{t \to \infty} \expect \bigl( v(t) \bigr)
    = \gamma \beta^{-1} \int_{\torus \times \real} \abs{\derivative{1}[\phi]{p}}^2 \, \d \mu
    = \int_{\torus \times \real} \phi \, p \, \d \mu =: D_{\phi}.
\]
so both $u(t)$ and $v(t)$ are asymptotically unbiased estimators of $D_{\phi}$.

\paragraph{Bias of the estimator.}%
In order to calculate the bias for finite times,
we assume stationary initial conditions and denote by $C_{\phi}(s) = \ip{p}{\e^{s \mathcal L} p}$ the velocity autocorrelation function.
Using It\^o's isometry, we have
\begin{align*}
    \expect \bigl(u(t)\bigr)
    &= \frac{1}{2t} \expect \left( \int_{0}^{t} p_s \, \d s \int_{0}^{t} p_u \, \d u \right)
    = \frac{1}{2t} \left( \int_{0}^{t} \int_{0}^{t} \expect (p_s p_u) \, \d s \, \d u \right) \\
    &= \frac{1}{2t} \left( \int_{0}^{t} \int_{0}^{t} C_{\phi}(\abs{s - u}) \, \d s \, \d u \right)
    =  \int_{0}^{t} C_{\phi}(s) \left(1 - \frac{s}{t}\right) \d s  \\
    &= D_{\phi} - \int_{0}^{\infty} \min\left(1, \frac{s}{t}\right) C_{\phi}(s) \, \d s.
\end{align*}
Since $\abs{C_{\phi}(s)} = \abs{\ip{p}{\e^{s \mathcal L} p}} \leq \norm{\e^{s \mathcal L} p} \leq c_1 \exp\bigl( - c_2 \min(\gamma, \gamma^{-1}) t\bigr)$, it holds
\[
    \forall t > 0, \quad \gamma > 0, \qquad
    \abs{\expect\bigl(u(t)\bigr) - D_{\phi}}
    \leq \frac{1}{t}\int_{0}^{\infty} s C_{\phi}(s) \, \d s
    \leq \frac{c_1}{c_2^2 t} \, \max\bigl(\gamma^{-2}, \gamma^2\bigr).
\]
It is well known that $D_{\phi} = \mathcal O(\gamma^{-1})$ in both the limit $\gamma \to 0$ and the limit $\gamma \to \infty$.
Therefore, the relative bias scales as $\max(\gamma^{-1}, \gamma^3) / t$.
Applying the same reasoning to $v(t)$, we obtain
\begin{align*}
    \expect \bigl(v(t)\bigr)
    &= D_{\psi} + \int_{0}^{t} \left(1 - \frac{s}{t}\right) \bigl( C_{\phi}(s) - C_{\psi}(s) \bigr) \, \d s. \\
    &= D_{\phi} - \int_{0}^{\infty} \min\left(1, \frac{s}{t}\right) \bigl( C_{\phi}(s) - C_{\psi}(s) \bigr) \, \d s,
\end{align*}
where $C_{\psi}(s) = \ip{\mathcal L \psi}{\e^{s \mathcal L} \mathcal L \psi}$.
We have
\begin{align*}
    C_{\phi}(s) - C_{\psi}(s)
    &= \ip{\mathcal L (\phi - \psi)}{\e^{s \mathcal L} \mathcal L \phi + \e^{s \mathcal L^*} \mathcal  L\psi} \\
    &\leq \norm{\mathcal L(\phi - \psi)}
    \left( \norm*{\e^{s \mathcal L}}[\mathcal B\left(L^2_0(\mu) \right)] \norm{\mathcal L \phi} + \norm*{\e^{s \mathcal L^*}}[\mathcal B\left(L^2_0(\mu) \right)] \norm{\mathcal L \psi} \right) \\
    &\leq c_1 \e^{-c_2 \min(\gamma, \gamma^{-1}) s} \norm{\mathcal L\psi + p}  \left(1 + \norm{\mathcal L \psi} \right),
\end{align*}
which leads to
\begin{align*}
    \forall t > 0, \quad \gamma > 0, \qquad
    \abs{\expect \left( v(t) \right) - D_{\phi}}
    \leq \frac{c_1}{c_2^2 t} \, \max(\gamma^2, \gamma^{-2}) \norm{\mathcal L\psi + p}  \left(1 + \norm{\mathcal L \psi} \right).
\end{align*}
The constant on the right-hand side is smaller when $\psi \approx \phi$,
and when $\psi = 0$ we recover the previous bound.

\paragraph{Variance of the estimator.}%
For the variance, we can obtain a crude bound using
\begin{align*}
    \mathrm{Var} (v(t))
    &\leq \expect \left( \abs{v(t) - D_{\psi}}^2 \right)
    = \frac{1}{4 t^2} \expect \big( |q_t - q_0 - \xi_t|^2 |q_t - q_0 + \xi_t|^2 \big) \\
    &\leq \frac{1}{4 t^2} \sqrt{\expect \left(  |q_t - q_0 - \xi_t|^4 \right)} \sqrt{\expect \left( |q_t - q_0 + \xi_t|^4 \right)}.
\end{align*}
We now bound
\begin{align*}
    \expect \left( |q_t - q_0 + \xi_t|^4 \right)
    &= \expect \left( \abs{\phi_t - \phi_0 - \psi_t + \psi_0 + I_{\psi} - I_{\phi}}^4 \right) \\
    &\leq 3^3 \left( \expect \left( \abs{\phi_t - \psi_t}^4 \right) + \expect \left( \abs{\phi_0 - \psi_0}^4 \right) + \expect \left( \abs{I_{\psi} - I_{\phi}}^4 \right) \right).
\end{align*}
The first two terms are bounded by $\norm{\phi - \psi}_{L^4(\mu)}^4$.
Using a moment inequality for It\^o integrals, the last term can be bounded as
\[
    \expect \left( \abs{I_{\psi} - I_{\phi}}^4 \right) \leq 6 t \expect \int_{0}^{t} \abs{\derivative{1}[\phi]{p}(q_s, p_s) - \derivative{1}[\psi]{p}(q_s, p_s)}^4 \, \d s = 6 t \int \abs{\derivative{1}[\phi]{p} - \derivative{1}[\psi]{p}}^4 \d \mu.
\]
Likewise, the other term can be bounded as
\begin{align*}
    \expect \left( |q_t - q_0 - \xi_t|^4 \right)
    &\leq 3^3 \left( \expect \left( 2 \norm{\phi + \psi}[L^4(\mu)]^4 + 6 t \norm{\derivative{1}[\phi]{p} + \derivative{1}[\psi]{p}}[L^{4}(\mu)]^4 \right) \right).
\end{align*}

\paragraph{Estimator diffusion coefficient.}%
\label{par:estimator_diffusion_coefficient}

Let $U_t = \sqrt{2 D} W_t$ and consider the estimator
\[
    2 \hat D = \frac{t_1 U_{t_1}^2 + \dotsb + t_N U_{t_N}^2}{t_1^2 + \dotsb + t_N^2},
\]
where $t_i = i \Delta_t$.
This estimator is clearly unbiased: $\expect (\hat D) = D$.
Let $\xi_i = (U_{t_{i}} - U_{t_{i-1}})/\sqrt{2D \Delta t}$, for $i = 1, \dotsc, N$.
It holds
\begin{align*}
    \frac{\hat D - D}{D}
    &= 6 \left( \frac{ \sum_{i=1}^{N} \sum_{j=i}^{N} (\xi_i ^2 - 1)  j  + 2 \sum_{i=1}^{N} \sum_{j=i+1}^{N} \sum_{k=i}^{j} \xi_i \xi_j k}
    {1 + \dotsb + N^2} \right) \\
    &= 3 \left( \frac{ \sum_{i=1}^{N} (\xi_i ^2 - 1)  (N+1-i)(N+i)  + 2 \sum_{i=1}^{N} \sum_{j=i+1}^{N} \xi_i \xi_j (j+1-i)(j+i)}
    { N(N+1)(2N+1)} \right).
\end{align*}
We calculate
\begin{align*}
    \expect \abs{ \frac{\hat D - D}{D} }^2
    &= 9 \left( \frac{ \sum_{i=1}^{N} \expect (\xi_i ^4 + 1 - 2 \xi_i^2)  (N+1-i)^2 (N+i)^2}
    { N^2 (N+1)^2 (2N+1)^2} \right) \\
    &\quad + 36 \left(  \frac{\sum_{i=1}^{N} \sum_{j=i+1}^{N} \expect \left( \abs{\xi_i}^2 \abs{\xi_j}^2 \right) \abs{j+1-i}^2 \abs{j+i}^2}
    { N^2 (N+1)^2 (2N+1)^2} \right) \\
    &= 18 \left( \frac{ \sum_{i=1}^{N}  (N+1-i)^2 (N+i)^2}
    { N^2 (N+1)^2 (2N+1)^2} \right)
    + 36 \left(  \frac{\sum_{i=1}^{N} \sum_{j=i+1}^{N} \abs{j+1-i}^2 \abs{j+i}^2}
    { N^2 (N+1)^2 (2N+1)^2} \right).
\end{align*}

\end{document}
