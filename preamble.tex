\usepackage[utf8]{inputenc}
\usepackage[margin=1in]{geometry}
\usepackage{color}
\usepackage{graphicx}
\usepackage{microtype}
\usepackage{array}
\usepackage{verbatim}
\usepackage{caption}
\usepackage{subcaption}
\usepackage{amsmath,amsthm,amsfonts,amssymb,latexsym}
\usepackage{bbm}
\usepackage{setspace}
\usepackage{xparse}
\usepackage{epstopdf}
\usepackage{pgf}
\usepackage[colorlinks=true,citecolor=blue]{hyperref}
\usepackage[nameinlink,capitalise]{cleveref}

\usepackage[style=trad-abbrv,doi=false,url=false,isbn=false,backend=biber]{biblatex}
\DeclareFieldFormat{volume}{volume \textbf{#1}}
\DeclareFieldFormat[article]{volume}{\textbf{#1}}
% \DeclareFieldFormat[book]{note}{}
% \DeclareFieldFormat[book]{pages}{}
\renewcommand*{\bibfont}{\footnotesize}
% \AtEveryBibitem{\clearfield{volume}\clearfield{issue}\clearfield{pages}\clearfield{series}\clearfield{number}}
\addbibresource{main.bib}

\usepackage{tikz}
\usepackage{tikz-cd}
\usepackage{pgfplotstable}
\pgfplotsset{compat=1.14}
\usetikzlibrary{patterns}
\usetikzlibrary{calc}
\usetikzlibrary{angles}
\usetikzlibrary{quotes}
\usetikzlibrary{external}

\onehalfspacing
% \setlength{\parskip}{6pt}

\DeclareDocumentCommand\abs{s m} {\IfBooleanTF{#1}{\left|#2\right|}{\left|#2\right|}}
\DeclareDocumentCommand\lp{m m o} {L^{#1}\left(#2 \IfNoValueF{#3}{,#3}\right)}
\DeclareDocumentCommand\norm{s m o} {\IfBooleanTF{#1}{\|#2\|}{\left\|#2\right\|}\IfNoValueF{#3}{_{#3}}}
\DeclareDocumentCommand\seminorm{m o o} {\left|#1\right|\IfNoValueF{#2}{_{#2 \IfNoValueF{#3}{,#3}}}}
\DeclareDocumentCommand\ip{s m m o} {\IfBooleanTF{#1}{\langle #2,#3 \rangle}{\left\langle #2,#3 \right\rangle}\IfNoValueF{#4}{_{#4}}}
\DeclareDocumentCommand\bigo{s o m} {\mathcal O\IfNoValueF{#2}{_{#2}}\IfBooleanTF{#1}{(#3)}{\left(#3\right)}}


\DeclareMathOperator*{\argmax}{arg\,max}
\DeclareMathOperator*{\argmin}{arg\,min}
\DeclareMathOperator*{\re}{Re}
\DeclareMathOperator*{\trace}{tr}
\DeclareMathOperator{\Span}{span}
\DeclareMathOperator{\sym}{sym}
\DeclareMathOperator{\sign}{sign}
\DeclareMathOperator{\diag}{diag}
\DeclareMathOperator{\id}{id}

% \DeclareMathOperator{\e}{e}
\newcommand{\e}{\mathrm{e}}
\newcommand{\revision}[1]{\textcolor{blue}{#1}}
\renewcommand{\revision}[1]{#1}
\newcommand{\gab}[1]{\textcolor{darkgreen}{#1}}
\newcommand{\commut}[2]{[#1, #2]}
\newcommand{\laplacian}{\Delta}
\newcommand{\correlation}[1]{\left< #1 \right>}
\newcommand{\dummy}{\,\cdot\,}
\newcommand{\expect}[0]{\mathbf{E}}
\newcommand{\proba}[0]{\mathbf{P}}
\newcommand{\var}[0]{\mathbf{V}}
\newcommand{\iip}[2]{\left(\!\left(#1, #2\right)\!\right)}
\newcommand{\nat}{\mathbf N}
\newcommand{\poly}{\mathbf P}
\newcommand{\real}{\mathbf R}
\newcommand{\integer}{\mathbf Z}
\newcommand{\torus}{\mathbf T}
\newcommand{\grad}{\nabla}
\newcommand{\imag}{\mathrm{i}}
\newcommand{\hess}{\nabla^2}
\newcommand{\vect}[1]{\boldsymbol{\mathbf #1}}
\newcommand{\mat}[1]{\vect #1}
\renewcommand{\det}[1]{\mathrm{det} \left( #1 \right)}
\renewcommand{\d}{\mathrm d}
\renewcommand{\t}{\mathsf T}
% \renewcommand{\t}{t}

\makeatletter
\DeclareDocumentCommand \derivative{s m o m}{%
    \def\@der{\IfBooleanTF{#1}{\mathrm{d}}{\partial}}
    \def\@default{%
        \mathchoice{%
                \frac{%
                    \@der\ifnum\pdfstrcmp{#2}{1}=0\else^{#2}\fi {\IfNoValueTF{#3}{}{#3}}
                }{%
                    \@for\@token:={#4}\do{\@der \@token}
                }
            } {%
                \@for\@token:={#4}\do{\@der_\@token} \IfNoValueTF{#3}{}{#3}
            } {} {}
    }
    \IfBooleanTF{#1}{\IfNoValueTF{#3}{\@default}{%
                #3%
                \ifnum\pdfstrcmp{#2}{1}=0'\else%
                \ifnum\pdfstrcmp{#2}{2}=0''\else%
                \ifnum\pdfstrcmp{#2}{3}=0^{(3)}\else%
                \ifnum\pdfstrcmp{#2}{4}=0^{(4)}\else%
                \ifnum\pdfstrcmp{#2}{5}=0^{(5)}\else%
                ^{(#2)}\fi\fi\fi\fi\fi
            }
        }{\@default}
}
\makeatother

\definecolor{darkred}{rgb}{.5,0,0}
\definecolor{darkgreen}{rgb}{0,.5,0}
\definecolor{darkblue}{rgb}{0,0,.5}
\newcommand{\red}[1]{\textcolor{darkred}{#1}}
\newcommand{\green}[1]{\textcolor{darkgreen}{#1}}

\theoremstyle{plain}
\newtheorem{assumption}{Assumption}[section]
\newtheorem{theorem}{Theorem}[section]
\newtheorem{lemma}[theorem]{Lemma}
\newtheorem{corollary}[theorem]{Corollary}
\newtheorem{proposition}[theorem]{Proposition}
\newtheorem{result}{Result}[section]
\newtheorem{remark}{Remark}[section]
\newtheorem{example}{Example}[section]
\numberwithin{equation}{section}

\newcounter{urbainCounter}
\newcommand{\urbain}[1]{\stepcounter{urbainCounter}\red{\arabic{urbainCounter}.} \green{#1}}
\crefname{equation}{}{}
\crefname{paragraph}{\S\!}{\S}
\crefname{figure}{Figure}{Figures}
% \crefname{section}{Section}{Sections}

\newcommand{\email}[1]{\href{#1}{#1}}
\newcommand{\orcidcolor}{ORC\textcolor{orcidlogocol}{ID}}
\newcommand{\orcid}[1]{\href{https://orcid.org/#1}{\includegraphics[width=.4cm]{figures/z_orcid.pdf}}}

%---------------- GABRIEL ------------
\usepackage{enumerate}
\newcommand{\eps}{\varepsilon}
\newcommand{\dps}{\displaystyle}
\newcommand{\cX}{\mathcal{X}}
\newcommand{\ri}{\mathrm{i}}
\renewcommand{\leq}{\leqslant}
\renewcommand{\geq}{\geqslant}
\renewcommand{\le}{\leqslant}
\renewcommand{\ge}{\geqslant}
% \usepackage{todonotes}
\usepackage{mathrsfs}

